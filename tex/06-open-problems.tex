\chapter{Open Problems}
We briefly discuss some important open problems which affect the design of languages for
MPC and obliviousness. These are not addressed by \lang and I feel that they are promising
areas for future research.

\paragraph{Resource Awareness}
MPC programs are expensive. Something something orders of magnitude slower~\cite{}. This due principally to the communication
required. For example in GMW, an AND gate requires a round of communication and the 1-4 OT requires an encryption scheme.
Programmers who are not experts in MPC will require information about the cost of the programs they are writing. This could be
handled elegantly by the language. There is ample research in type-based resource analysis~\cite{}, for example. \ins{TODO: this is rough}

\paragraph{\lang Compiler}
Evaluating a sophisticated compiler for \lang against other MPC frameworks such as Obliv-C and ABY is left to future work. We conjecture
that such an implementation would be competitive with state-of-the-art MPC languages, while also providing strong safety and security
guarantees.

\paragraph{Oblivious Data Structures}
Our prior work has shown that Oblivious Data Structures (ODS's), proposed by Wang et al., do not satisfy PMTO.\ins{replace with ref. to theorem}
Even though ODS's are safe due to negligible overflow probability, we are unable to verify that fact in \lang. We would like to find some
way to support ODS's within the language while retaining the security guarantees provided by the type system.

\paragraph{Other Security Policies}
PMTO\% is perhaps the most obvious security policy one could prove about MPC programs which admit ORAM. However, it is not the only property
one could prove. There is a rich literature of different declassification policies, as well as weaker variants of (P)MTO like Differential
MTO~\cite{} and manual computational proofs of securit with explicit complexity-theoretic reductions~\cite{easycrypt}.

\paragraph{Usability of Abstractly Sequential Languages}
We claim in this proposal that abstractly sequential languages are a good choice for MPC. We justify that claim by arguing that
programmers need not think about the distributed deployment of their program. Furthermore, the type safety theorem for the sequential
semantics implies safety of the distributed semantics. However, a more empirical approach to the question of usability is warranted.
A user study comparing two languages, one of which exposes more of the distributed computation, would lend more credibility to our claim
that abstractly sequential languages are preferable.

\paragraph{Support for PIR-based ORAM}
The most efficient modern ORAM schemes in the MPC context use Private Information Retrieval (PIR) protocols. These protocols rely on
features such as function secret sharing and cryptographic pseudo-random functions. These more sophisticated cryptographic protocols
are not in the current scope of \lang.
