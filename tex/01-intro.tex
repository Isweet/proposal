\chapter{Introduction}
\label{ch:intro}

Programming languages have a duty of \emph{confidentiality}. They are responsible for protecting the
data that we designate as sensitive or private. Most modern computation is \emph{mutual} --- computation
is performed over the private information of many parties --- which means that programming languages
have a duty of \emph{mutual confidentiality}. What do languages that guarantee mutual confidentiality look
like? How can they be improved?

\section{Secure Multiparty Computation and Obliviousness}
\label{sec:intro-mpc-obliv}

% What is MPC, in a nutshell?
\emph{Secure Multiparty Computation} (MPC) is a subfield of cryptography
that allows mutually untrusting parties to compute arbitrary functions over their private inputs while revealing nothing
except the function output. That is, MPC allows parties to work together to run programs \emph{under encryption}.

% Problem: MPC programs are hard to write and read for novices.
% Solution: Abstract Sequentiality.
Approaches to MPC have improved significantly over the years. The first full implementation, FairPlay~\cite{USENIX:MNPS04}, could evaluate
only a few hundred Boolean gates per second. Modern implementations with custom deployments evaluate \emph{billions} of Boolean gates per
second~\cite{CCS:AFLNO16}, and cryptographers continue to reduce the cost of MPC. Despite increasing efficiency and compelling applications
(secure auctions, secure databases, collaborative machine learning, and any other imaginable application with security concerns),
MPC has not been widely adopted. One significant barrier to adoption is a lack of appropriate infrastructure. Today, it is difficult
for non-experts to understand and work in the complex distributed model that MPC requires. To assuage this difficulty, some recent
MPC lanuages (e.g. SCVM~\cite{liu14scram}, Wysteria~\cite{rastogi14wysteria}, and \system~\cite{symphony22}) are
\textbf{Abstractly Sequential}. An \textbf{Abstractly Sequential} language hides the parallel deployment of the MPC
program from the programmer. This proposal builds heavily on \system, which is discussed formally in~\Cref{ch:mpc}.

% Problem: MPC programs are slow, linear ORAM is intractable.
% Solution: Uniform, random sampling => logarithmic ORAM.

In addition to a lack of infrastructure, MPC languages also fail to provide adequate support for RAM-model programming.
The standard means by which RAM-model computation is supported in MPC languages is using \emph{Oblivious RAM} (ORAM). To date, most MPC languages
simply do not provide any access to ORAM. In their 2019 SoK,~\citet{HastingsHNZ19} observe that only a ``few frameworks have ORAM support, either
natively (ObliVM and SCALE-MAMBA) or via a library (Obliv-C).'' The three languages mentioned, ObliVM~\cite{oblivm}, SCALE-MAMBA~\cite{SCALEMAMBA}, and
Obliv-C~\cite{cryptoeprint:2015:1153}, are \emph{probabilistic} MPC languages. According to~\citet{G87}, a language must have support for uniform, random sampling
to implement asymptotically optimal ORAM. Indeed, each of these languages provides a highly optimized implementation of
Circuit ORAM~\cite{10.1145/2810103.2813634}, which is known to be asymptotically optimal (i.e. $O(\log{n})$).

% Problem: MPC programs that use uniform, random sampling for efficiency are not guaranteed to be safe (confidential).
% Solution: High assurance programs via an oblivious or leak-free declassification which is checked for confidentiality.
It is common practice in probabilistic MPC languages to declassify values with the expectation that they
do not reveal any information. For example, consider the One-Time Pad (OTP) in~\Cref{fig:mpc-otp}.

\begin{figure}[h]
M⁅
\begin{array}{r@{␠}lcl}
     «0:» & 𝑚𝑐3l{ ⦑par⦒[\alice,\bob] }
  \\ «1:» & ␠⦑let⦒␣k ⧼=⧽ ⦑par⦒[\alice]␣⦑unif⦒⸢ρ⸣␣⦑bool⦒␣⦑in⦒
  \\ «2:» & ␠⦑let⦒␣s ⧼=⧽ ⦑par⦒[\alice]␣⦑read⦒␣⦑bool⦒␣⦑in⦒
  \\ «3:» & ␠⦑let⦒␣e ⧼=⧽ ⦑send⦒[\alice → \alice,\bob]␣⦑par⦒[\alice]␣k ⊕ s␣⦑in⦒
  \\ «4:» & ␠…
\end{array}
M⁆
\caption{\lang code for a One-Time Pad.}
\label{fig:mpc-otp}
\end{figure}

Line 0 indicates that this is a joint computation among \alice and \bob. On line 1, \alice generates a uniform, random bit which will serve
as her encryption key. On line 2, she reads a secret bit from local storage. Finally, on line 3, she encrypts her secret, ⸨s⸩, by XORing it
with the key ⸨k⸩ and then sends the result to \bob.~\footnote{We write ⸨⦑send⦒⸩ as syntactic sugar for a ⸨⦑share⦒⸩ followed immediately by
a ⸨⦑reveal⦒⸤⦑true⦒⸥⸩.} Does \bob learn any information about \alice's secret? He doesn't, because it has been encrypted
by the random bit ⸨k⸩ which he doesn't know. As long as ⸨k⸩ is never revealed to \bob, \alice's secret is safe.
However, in existing MPC languages, there is no way to specify that the ⸨⦑send⦒⸩ on line 3 reveals no information about ⸨s⸩.
A \emph{high assurance} language ensures that programs are probabilistically oblivious. Such languages give programmers the ability to specify
that probabilistic values may be safely revealed to other parties. They give novices the opportunity to experiment safely, and free experts from having to
manually audit their programs.

In summary, we would like a modern MPC language to satisfy the following properties.

\begin{enumerate}
\item \label{itm:sequential} \textbf{\underline{Abstractly Sequential}} -- An abstractly sequential language, which hides the parallel deployment from the
  programmer, is necessary for lowering the barrier of entry to MPC languages. Being able to program as though the program is being executed sequentially,
  without worrying about the parallel deployment of the program, significantly reduces the complexity of MPC programming.
\item \label{itm:probabilistic} \textbf{\underline{Probabilistic}} -- A probabilistic language, which allows sampling from uniform, random distributions,
  is necessary for implementing asymptotically optimal ORAM, oblivious algorithms, and oblivious data structures. When more efficient oblivious protocols are
  invented (which happens every year), cryptographers would like to be able to implement these protocols as libraries in MPC languages.
\item \label{itm:assurance} \textbf{\underline{High Assurance}} -- A high assurance language, which guarantees that all programs are oblivious, is necessary
  for certifying that programs do not leak information through side channels. Without certification of obliviousness, a domain expert
  would need to manually audit the declassifications in the program which are expected to reveal no information. Such declassifications
  are common, for example, in tree-based ORAM constructions.
\end{enumerate}

\paragraph{Problem.} No existing MPC language is (all together) \textbf{Abstractly Sequential}, \textbf{Probabilistic}, and \textbf{High Assurance}.

\section{Proposed Work}
\label{sec:intro-proposal}

\paragraph{Hypothesis.} It is possible to design and implement an efficient language for MPC which is \textbf{Abstractly Sequential},
\textbf{Probabilistic}, and \textbf{High Assurance} (as defined above).

\paragraph{Contribution: \mpc is MTO\%.} In ongoing work, we show that \mpc's type system guarantees that MPC programs are
\emph{Memory Trace Oblivious Modulo Declassifications} (MTO\%). To our knowledge, this is the first time such a theorem has been
proved for an MPC language. The SCVM language~\cite{liu14scram} has a proof of a related property based on crypto-style ``simulation.''

\paragraph{Contribution: \mpc Performance Measurement.} In already completed work, we implemented \system, a Haskell
interpreter for \mpc. \system supports both the sequential and distributed semantics with an MPC backend based on EMP~\cite{emp-toolkit}.
We performed an empirical evaluation of the runtime performance of \system by comparing to Obliv-C on five standard MPC benchmarks. We concluded
that \system is slower than Obliv-C, but in the same ballpark. Our results strongly suggest that \system could be made competitive with Obliv-C
with standard optimizations (e.g. compiling rather than interpreting).

\paragraph{Contribution: A Type System for \mpc.} We will extend \mpc (and \system) with a sound static type system. With Simulation,
type soundness guarantees that well-typed programs evaluate to the same result in both the sequential and distributed semantics.
This also implies that well-typed programs can't enounter distributed programming errors such as hanging and deadlock.

\paragraph{Contribution: \obliv is PMTO.} In already completed work, we show that ObliVM's~\cite{oblivm} type system is not oblivious,
and show how to fix it so that it satisfies \emph{Probabilistic Memory Trace Obliviousness} (PMTO). We prove this property for \obliv,
a non-MPC language for oblivious programming.

\paragraph{Contribution: \obliv Expressiveness Measurement.} In already completed work, we implemented \textsc{OblivML}, an OCaml
interpreter for \obliv. \textsc{OblivML} supports type-checking and interpretation of \obliv programs. We performed an empirical
evaluation of \textsc{OblivML}, showing that it is adequately expressive by type-checking oblivious algorithms such as Tree ORAM.
We also discovered that Oblivius Stacks are not PMTO~\cite{sweet20odsnotpmto}.

\paragraph{Contribution: Identifying PMTO\%.} In ongoing work, we identify and define \emph{Probabilistic Memory Trace Obliviousness Modulo Declassification}
(PMTO\%) as an appropriate definition of security for probabilistic MPC languages. We also investigate and explicate the connection between
this property and other common security properties involving declassifications (such as Gradual Release~\cite{gradual-release}).

\paragraph{Contribution: \lang, an \textbf{Abstractly Sequential}, \textbf{Probabilistic}, \textbf{High Assurance} MPC language.}
We will design, implement, and evaluate the expressiveness and efficiency of \lang, an MPC language satisfying the three desirable
properties above. To do so, we will extend \mpc with expressions for uniform, random sampling and declassification. This satisfies
Property~\ref{itm:probabilistic} above. Next we will design the type system, drawing from previous work on Wysteria~\cite{rastogi14wysteria}
and \obliv~\cite{darais20obliv}, and prove that \lang satisfies type safety, simulation, and obliviousness theorems.
The type safety and simulation theorems establish that \lang satisfies Property~\ref{itm:sequential}. The type safety and obliviousness
theorems establish that \lang satisfies Property~\ref{itm:assurance}. We will provide evidence that \lang is expressive and efficient
by evaluating \lang against four benchmarks: One-Time Pad, Circuit ORAM, Binary Search (on Circuit ORAM), and Optimized QuickSort~\cite{obliv-quicksort}.
Expressiveness of \lang is established by type checking the benchmarks, while efficiency is evaluated by measuring execution time and
gate counts and comparing against Obliv-C. Having accomplished all this, we will have confirmed our hypothesis.

\Cref{tab:timeline} describes the remaining tasks and gives a rough estimate for their time to completion. Tasks
in \colorbox{implColor}{red} require adding features to the existing Haskell interpreter for \mpc. Tasks in
\colorbox{theoryColor}{blue} require adaptation of the formal metatheory of \mpc.

\begin{table*}
  \centering
  \small

\begin{tabular}{|p{.33\textwidth}|p{.33\textwidth}|p{.33\textwidth}|}
  \hline
  \textbf{Task} & \textbf{Description} & \textbf{Estimated Time of Completion} \\
  \hline

  \rowcolor{implColor}
  Probabilistic Choice &
  Implement probabilistic choice and declassification for \lang. &
  \ins{TODO} \\ \hline

  \rowcolor{implColor}
  Type Checker &
  Implement the type checker for \lang. &
  \ins{TODO} \\ \hline

  \rowcolor{implColor}
  Case Studies:
  \begin{itemize}
  \item One-Time Pad
  \item Circuit ORAM
  \item Binary Search
  \item Optimized QuickSort
  \end{itemize} &
  Implement the ORAM case studies in \lang and benchmark their performance. &
  \ins{TODO} \\ \hline

  \rowcolor{theoryColor}
  Static Semantics &
  Design the type system of \lang. &
  \ins{TODO} \\ \hline

  \rowcolor{theoryColor}
  Type Soundness Proof &
  Prove \lang satisfies Type Soundness. &
  \ins{TODO} \\ \hline

  \rowcolor{theoryColor}
  Simulation Proof &
  Prove \lang satisfies Simulation. &
  \ins{TODO} \\ \hline

  \rowcolor{theoryColor}
  PMTO\% Proof &
  Prove \lang satisfies PMTO\%. &
  \ins{TODO} \\ \hline

  Thesis Writing &
  Write and defend the dissertation. &
  \ins{TODO} \\ \hline
\end{tabular}
  \caption{
    Description of remaining tasks and an estimate of their time to completion.
    }
\label{tab:timeline}
\end{table*}
