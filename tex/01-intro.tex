\chapter{Introduction}
\label{ch:intro}

\epigraph{It seems like everywhere I go \\
          The more I see, the less I know}{
            Michael Franti \& Spearhead \\
            \emph{Say Hey I Love You}}

If our adversaries were more like Michael Franti then we wouldn't need to concern ourselves with privacy.
Our programming languages have a duty of \emph{confidentiality}. They are responsible for protecting the
data that we designate as sensitive or private. Most computation today is \emph{mutual} --- computation
is performed over the private information of many parties --- which means that programming languages really
have a duty of \emph{mutual confidentiality}. What do languages that guarantee mutual confidentiality look
like? How can they be improved?

\section{Secure Multiparty Computation and Obliviousness}
\label{sec:intro}

% What is MPC, in a nutshell?
\emph{Secure Multiparty Computation} (MPC) is a subfield of cryptography
that allows mutually untrusting parties to compute arbitrary functions over their private inputs while revealing nothing
except the function output. That is, MPC allows parties to work together to run programs \emph{under encryption}.

% Problem: MPC programs are hard to write and read for novices.
% Solution: Abstract Sequentiality.
Approaches to MPC have improved significantly over the years. The first full implementation, FairPlay~\cite{todo}, could evaluate
only a few hundred Boolean gates per second. Modern implementations on custom setups evaluate \emph{billions} of Boolean gates per
second~\cite{todo}, and cryptographers continue to reduce the cost of MPC. Despite increasing efficiency and compelling applications
(secure auctions, secure databases, collaborative machine learning, and any other imaginable application with security concerns),
MPC has not been widely adopted. One significant barrier to adoption is a lack of appropriate infrastructure. Today, it is difficult
for non-experts to understand and work in the complex distributed model that MPC requires. To assuage this difficulty, some recent
MPC lanuages (e.g. SCVM, Wysteria, and Symphony) are \textbf{abstractly sequential}. An \textbf{abstractly sequential} language hides
the parallel deployment of the MPC program from the programmer. This proposal builds heavily on one such language, Symphony, which is
discussed formally in~\cref{sec:lam-mpc}.

% Problem: MPC programs are slow, linear ORAM is intractable.
% Solution: Uniform, random sampling => logarithmic ORAM.

In addition to a lack of infrastructure, MPC languages also fail to provide adequate support for RAM-model programming.
The standard means by which RAM-model computation is supported in MPC languages is using \emph{Oblivious RAM} (ORAM). To date, most MPC languages
simply do not provide any access to ORAM. In their 2019 SoK,~\citet{todo} observe that only a ``few frameworks have ORAM support, either
natively (ObliVM and SCALE-MAMBA) or via a library (Obliv-C).'' The three languages mentioned, ObliVM~\cite{todo}, SCALE-MAMBA~\cite{todo}, and
Obliv-C~\cite{todo}, are \textbf{probabilistic} MPC languages. According to~\citet{todo}, a language must have support for uniform, random sampling
to implement asymptotically optimal ORAM. Indeed, each of these languages provides a highly optimized implementation of
Circuit ORAM~\cite{todo}, which is known to be asymptotically optimal (i.e. $O(\log{n})$).

% Problem: MPC programs that use uniform, random sampling for efficiency are not guaranteed to be safe (confidential).
% Solution: High assurance programs via an oblivious or leak-free declassification which is checked for confidentiality.
It is common practice in probabilistic MPC languages to declassify values with the expectation that they
do not reveal any information. For example, consider the One-Time Pad (OTP) in~\cref{fig:otp-symphony}.

\begin{figure}[h]
M⁅
\begin{array}{r@{␠}lcl}
     «0:» & 𝑚𝑐3l{ ⦑par⦒[\alice,\bob] }
  \\ «1:» & ␠⦑let⦒␣k ⧼=⧽ ⦑par⦒[\alice]␣𝒰(❴0,1❵)␣⦑in⦒
  \\ «2:» & ␠⦑let⦒␣s ⧼=⧽ ⦑par⦒[\alice]␣⦑read⦒␣⦑in⦒
  \\ «3:» & ␠⦑let⦒␣e ⧼=⧽ ⦑declassify⦒[\alice → \alice,\bob]␣⦑par⦒[\alice]␣k ⊕ s␣⦑in⦒
  \\ «4:» & ␠…
\end{array}
M⁆
\caption{\mpc{} code for a One-Time Pad}
\label{fig:otp-symphony}
\end{figure}

This example is a computation between two parties, \alice and \bob. On line 1, \alice generates a local uniform, random bit which will serve
as her encryption key. On line 2, she reads a secret bit from local storage. Finally, on line 3, she encrypts her secret, ⸨s⸩, by XORing it
with the key ⸨k⸩ and then declassifies the result to \bob. Does \bob learn any information about \alice's secret? He doesn't, because it has been encrypted
by the random bit ⸨k⸩ which he doesn't know. As long as ⸨k⸩ is never revealed to \bob, \alice's secret is safe.
However, in existing MPC languages, there is no way to specify that the declassification on line 3 reveals no information about ⸨s⸩.
A \textbf{high assurance} language ensures that programs are probabilistically oblivious. They give programmers the ability to specify
that probabilistic values may be safely revealed to other parties. They gives novices the opportunity to experiment safely, and free experts from having to
manually audit their programs.

In summary, we would like the following properties from our MPC language.

\begin{enumerate}
\item \label{itm:sequential} \textbf{\underline{Abstractly Sequential} -- An abstractly sequential language, which hides the parallel deployment from the
  programmer, is necessary for lowering the barrier of entry to MPC languages.}
  Being able to program as though the program is being executed sequentially, without worrying about the parallel deployment
  of the program, significantly reduces the complexity of MPC programming.
\item \label{itm:probabilistic} \textbf{\underline{Probabilistic} -- A probabilistic language, which allows sampling from uniform, random distributions, is necessary
  for implementing asymptotically optimal ORAM, oblivious algorithms, and data structures.}
  When more efficient oblivious protocols are invented (which happens every year), cryptographers would like to be able to
  implement these protocols as libraries in MPC languages.
\item \label{itm:assurance} \textbf{\underline{High Assurance} -- A high assurance language, which guarantees that all programs are oblivious, is necessary
  for certifying that programs do not leak information through side channels.} Without certification of obliviousness, a domain expert
  would need to manually audit the declassifications in the program which are expected to reveal no information. Such declassifications
  are common in, for example, tree-based ORAM constructions.
\end{enumerate}

\paragraph{Problem.} No existing MPC language is \textbf{(all together) abstractly sequential, probabilistic, and high assurance.}

\section{Proposed Work}
\label{sec:intro-proposal}

\paragraph{Hypothesis.} It is possible to design and implement a language for MPC which is \textbf{abstractly sequential, probabilistic,
  and high assurance (as defined above).}

\paragraph{Contribution: \obliv is PMTO.} In prior work, we show that ObliVM's~\cite{todo} type system is not oblivious,
and show how to fix it in such a way that is satisfies \emph{Probabilistic Memory Trace Obliviousness} (PMTO).
We prove this property for \obliv, a non-MPC language for oblivious programming.

\paragraph{Contribution: \mpc is MTO\%.} In ongoing work, we show that Symphony's type system guarantees that MPC programs are
\emph{Memory Trace Oblivious Modulo Declassifications} (MTO\%). To our knowledge, this is the first time such a theorem has been
proved for an MPC language. The SCVM language has a proof of a related property based on crypto-style ``simulation.''

\paragraph{Contribution: Identifying PMTO\%.} We identify and define \emph{Probabilistic Memory Trace Oblivious Modulo Declassification}
(PMTO\%) as an appropriate definition of security for probabilistic MPC languages. We also investigate and explicate the connection between
this property and other common security properties involving declassifications (such as Gradual Release~\cite{todo}).

The contributions described above have already been accomplished. They are necessary but not sufficient to confirm our hypothesis.
To confirm our hypothesis, we propose the design and implementation of a new language for MPC, called \lang, which extends the
Symphony language with primitives for drawing uniform, random samples and declassifying them.

\paragraph{Contribution: \lang, an abstractly sequential, probabilistic, and high assurance MPC language.}
\lang will extend Symphony with uniform, random sampling. This will immediately satisfy
Property~\ref{itm:probabilistic} above. We will provide evidence that this property is useful by
implementing various ORAM protocols (Trivial ORAM, Circuit ORAM, and Recursive (Circuit) ORAM). We will show through
an empirical evaluation that these case studies have the expected asymptotic behavior. Next, we will prove that \lang satisfies
the type safety and simulation theorems by adapting the proofs of these properties for Symphony. This will establish
that \lang satisfies Property~\ref{itm:sequential}.
Finally, we will design the type system of \lang to enforce PMTO\% and confirm that the case studies (ORAM protocols) are well-typed.
This will establish that \lang satisfies Property~\ref{itm:assurance} while also admitting interesting protocols. Having
accomplished everything above, we will have confirmed our hypothesis.

We summarize the tasks involved in the last contribution above, and give a rough timeline for their completion. Tasks in
\colorbox{implColor}{red} require adding additional functionality to the existing Haskell implementation of Symphony.
Tasks in \colorbox{theoryColor}{blue} require adapting the formal metatheory of Symphony. \\

\ins{TODO: when proposal document is done, revisit this and update timeline.}

\begin{tabular}{|p{.30\textwidth}|p{.30\textwidth}|p{.30\textwidth}|}
  \hline
  \textbf{Task} & \textbf{Description} & \textbf{Estimated Time of Completion} \\
  \hline
  \rowcolor{implColor}
  Case Studies:
  \begin{itemize}
  \item Trivial ORAM
  \item Tree ORAM(s)
  \item Recursive ORAM
  \end{itemize}    & Implement case studies in \lang;
  show that they are functionally correct and typecheck & 10/07/2020 ($\sim$2 weeks)  \\ \hline
  \rowcolor{implColor}
  Type Checker     & Implement the type checker for \lang                  & 11/07/2020 ($\sim$1 month)  \\ \hline
  \rowcolor{implColor}
  MPC Interpreter  & Implement an EMP MPC backend for \lang                & 01/07/2021 ($\sim$2 months) \\ \hline
  \rowcolor{theoryColor}
  Static Semantics & \lang is \mpc + \obliv                                & 03/07/2021 ($\sim$2 months) \\ \hline
  \rowcolor{theoryColor}
  PMTO\%           & Prove \lang satisfies PMTO\%                          & 05/07/2021 ($\sim$2 months) \\ \hline
  \rowcolor{theoryColor}
  Type Soundness   & Prove \lang satisfies Type Soundness                  & 05/14/2021 ($\sim$1 week)   \\ \hline
  \rowcolor{theoryColor}
  Simulation       & Prove \lang satisfies Simulation                      & 05/21/2021 ($\sim$1 week)   \\ \hline
  Thesis           & Write the thesis and defend it                        & 10/21/2021 ($\sim$5 months) \\ \hline
\end{tabular}

\chapter{Background}
\label{ch:background}

\epigraph{Trust in me in all you do \\
          Have the faith I have in you \\
          Love will see us through \\
          If only you trust in me}{
            Etta James \\
            \emph{Trust in Me}}

We begin by introducing the Symphony language using the Millionaire's Problem. Having done so, we will expand on the MPC language
properties that we advocate for in Section~\ref{sec:background-properties}.
Readers unfamiliar with MPC are encouraged to read Appendix~\ref{ch:gmw}, which describes the GMW protocol.
Readers unfamiliar with ORAM are encouraged to read Appendix~\ref{ch:oram}, which describes the ORAM protocols relevant to this proposal.

\section{A Taste of \mpc}
\label{sec:background-symphony}

From here on, we instead refer to Symphony as \mpc to emphasize the similarities with \obliv which is presented in
Chapter~\ref{ch:lam-obliv}. As a means of gentle introduction, let's consider the Millionaire's Problem in which two wealthy parties,
\alice and \bob, would like to know who is wealthier.

\begin{figure}[h]
M⁅
\begin{array}{r@{␠}lcl}
   «0:» & 𝑚𝑐3l{ ⦑par⦒[\alice,\bob] }
\\ «1:» & ␠⦑let⦒␣x    ⧼=⧽ ⦑par⦒[\alice]␣⦑read⦒␣⦑in⦒
\\ «2:» & ␠⦑let⦒␣y    ⧼=⧽ ⦑par⦒[\bob]␣⦑read⦒␣⦑in⦒
\\ «3:» & ␠⦑let⦒␣sx   ⧼=⧽ ⦑share⦒[\alice → \alice,\bob]␣x␣⦑in⦒
\\ «4:» & ␠⦑let⦒␣sy   ⧼=⧽ ⦑share⦒[\bob → \alice,\bob]␣y␣⦑in⦒
\\ «5:» & ␠⦑let⦒␣r    ⧼=⧽ sx < sy␣⦑in⦒
\\ «6:» & ␠⦑let⦒␣z    ⧼=⧽ ⦑reveal⦒[\alice,\bob]␣r␣⦑in⦒
\\ «7:» & ␠…
\end{array}
M⁆
\caption{\mpc{} code for the Millionaire's Problem}
\label{fig:millionaires-symphony}
\end{figure}

The code in Figure~\ref{fig:millionaires-symphony} is written in the \mpc language. On line 0, the ⸨⦑par⦒[\alice,\bob]⸩ block
indicates that both \alice and \bob will execute everything in the block's lexical scope. Any other parties will ignore the
contents of the block and generate an \emph{opaque value} denoted \opaque. On line 1, \alice ⸨⦑read⦒⸩s her own wealth,
⸨\alices{⋖wealth⋗}⸩, from local storage and binds it to ⸨x⸩. \bob, however, evaluates the expression ⸨⦑par⦒[\alice]␣⦑read⦒⸩ to
\opaque because he is not included in the ⸨⦑par⦒[\alice]⸩ block. On line 2, the same thing happens except \bob reads his wealth,
⸨\bobs{⋖wealth⋗}⸩, and binds it to ⸨y⸩ while \alice binds \opaque to ⸨y⸩. In summary, after line 2, the local environemnts are:

M⁅
  Aːcc
  A⁅ \alice & \bob
  A⁃ ⟨x ↦ \alices{⋖wealth⋗},␣y ↦ \opaque⟩␠ & ␠⟨x ↦ \opaque,␣y ↦ \bobs{⋖wealth⋗}⟩
  A⁆
M⁆

On line 3, \alice splits her ⸨\alices{⋖wealth⋗}⸩ into two shares. She keeps her share and sends \bob's share to him. We use the same
notation for shares which appears in Appendix~\ref{ch:gmw} to emphasize the connection between the explanation of GMW and the execution model
of \mpc. On line 4, \bob splits his ⸨\bobs{⋖wealth⋗}⸩ into two shares. He keeps his share and send's \alice's share to her. So, after
line 4 the local environments are:

M⁅
  Aːllll
  A⁅ 𝑚𝑐2c{\alice} & 𝑚𝑐2c{\bob}
    A⁃ ⟨…,&␣sx ↦ \aliceSh{\alices{⋖wealth⋗}}, & ␠⟨…,&␣sx ↦ \bobSh{\alices{⋖wealth⋗}},
    A⁃    &␣sy ↦ \aliceSh{\bobs{⋖wealth⋗}}⟩   &     &␣sy ↦ \bobSh{\bobs{⋖wealth⋗}}⟩
  A⁆
M⁆

On line 5, \alice and \bob compute over their shares to produce a share indicating who is wealthier. Here we assume that the language knows
how to compute ⸨<⸩ over shares. For example, if the underlying MPC protocol were GMW then we would encode ⸨<⸩ as a magnitude circuit made
up of XOR and AND gates.

M⁅
  Aːcc
  A⁅ \alice & \bob
  A⁃ ⟨…,␣r ↦ \aliceSh{A⸨\alices{⋖wealth⋗} < \bobs{⋖wealth⋗}A⸩}⟩ & ␠⟨…,␣r ↦ \bobSh{A⸨\alices{⋖wealth⋗} < \bobs{⋖wealth⋗}A⸩}⟩
  A⁆
M⁆

Finally, on line 6, the shares of ⸨\alices{⋖wealth⋗} < \bobs{⋖wealth⋗}⸩ among \alice and \bob are combined and the
result is revealed to both parties. Note that we could have revealed the result only to \alice, for example,
in which case \bob would send his share to \alice but not vice versa. Then, \alice could recover the cleartext result but \bob could not.
In this case, however, they both send their shares to each other. The XOR operator, ⸨⊕⸩, is used to combine shares implicitly as part
of the ⸨⦑reveal⦒⸩.

M⁅
  Aːllll
  A⁅ 𝑚𝑐2c{\alice} & 𝑚𝑐2c{\bob}
    A⁃ ⟨…,␣z &{} ↦ \aliceSh{A⸨\alices{⋖wealth⋗} < \bobs{⋖wealth⋗}A⸩} & ␠⟨…,␣z &{} ↦ \aliceSh{A⸨\alices{⋖wealth⋗} < \bobs{⋖wealth⋗}A⸩}
    A⁃       &{}␣⊕ \bobSh{A⸨\alices{⋖wealth⋗} < \bobs{⋖wealth⋗}A⸩} & &{}␣⊕ \bobSh{A⸨\alices{⋖wealth⋗} < \bobs{⋖wealth⋗}A⸩}
    A⁃       &{}␣= \alices{⋖wealth⋗} < \bobs{⋖wealth⋗}⟩ & &{}␣= \alices{⋖wealth⋗} < \bobs{⋖wealth⋗}⟩
  A⁆
M⁆

There are some interesting things to notice about this program (and \mpc more generally). First, the programmer can control which parties
evaluate which expressions by using ⸨⦑par⦒⸩ blocks. Second, cleartext ⸨⦑read⦒⸩ operations (lines 1,2) are mixed with ciphertext operations
(lines 3,4,5,6). Lastly, the language is \textbf{abstractly sequential}. In a deployment, this program would run
independently on \alice and \bob who would communicate to jointly compute the result. Our explanation of the execution of
Figure~\ref{fig:millionaires-symphony} highlights the intuition of abstract sequentiality.

\section{Desirable Properties of MPC}
\label{sec:background-properties}

In Section~\ref{sec:intro} we claimed that a MPC language ought to be \textbf{abstractly sequential}, \textbf{probabilistic},
and \textbf{high assurance}. But, what are these properties and why do we care about them?

\subsection{Abstractly Sequential MPC Languages}
\label{subsec:background-properties-centralized}

A language which is abstractly sequential liberates the programmer from managing a parallel computation. In the context of MPC,
the programmer does not need to perform commands conditioned on which party is executing. To see the difference, contrast the
Obliv-C implementation of the Millionaire's Problem in Figure~\ref{fig:millionaires-oblivc} with the \mpc implementation in
Figure~\ref{fig:millionaires-symphony}.

\begin{figure}[h]
\begin{lstlisting}[language=c,basicstyle=\footnotesize\ttfamily,numbers=left,stepnumber=1]
  // File: million.h
  typedef struct {
    int in;
    bool out;
  } ProtocolIO;

  void millionaire (void *args);

  // File: million.oc
  #include <million.h>
  #include <obliv.oh>

  void millionaire (void *args) {
    ProtocolIO *io = args;
    obliv int a, b;
    obliv bool res = false;
    a = feedOblivInt (io->in, 1);
    b = feedOblivInt (io->in, 2);
    obliv if (a < b) res = true;
    revealOblivBool(&io->out, res, 0);
  }

  // File: million.c
  #include <million.h>

  int main (int argc, char *argv[]) {
    ProtocolDesc pd;
    ProtocolIO io;
    int p = (argv[1] == '1' ? 1 : 2);
    sscanf(argv[2], "%d", &io.in);
    // ... set up TCP connections

    setCurrentParty(&pd, p);
    execYaoProtocol(&pd, millionaire, &io);
    printf ("%d\n", io.out);
    // ... cleanup
  }
\end{lstlisting}
\caption{Obliv-C code for the Millionaire's Problem}
\label{fig:millionaires-oblivc}
\end{figure}

We can see that many of the details of the parallel, MPC deployment have leaked into the application code. The
\lstinline[language=c,basicstyle=\ttfamily]{main} function is responsible for all sorts of scaffolding such as establishing TCP connections (line 31),
explicitly executing the protocol (line 34), and populating the \lstinline[language=c,basicstyle=\ttfamily]{io} variable with party inputs (line 30). As
a point of comparison, take a look back at Figure~\ref{fig:millionaires-symphony}. Notice that this code makes no mention of TCP connections,
explicit execution of any protocol, or eagerly reading and caching secret inputs.

In Chapter~\ref{ch:lam-mpc} we will formally define an \textbf{abstractly sequential} MPC language as one which satisfies two theorems:
forward simulation and type safety. Forward simulation guarantees that if the sequential interpretation of a program doesn't get stuck,
then neither does its parallel deployment. Type safety ensures that well-typed programs do not get stuck in the sequential interpretation.
These two theorems work together to guarantee that well-typed programs cannot encounter, for example, a liveness error in which one party
fails to send a message that another party is blocked on.

Although we will not make any formal claims regarding the usability of the language, we claim that an MPC language which hides the parallel
nature of MPC is desirable.

\subsection{Probabilistic MPC Languages}
\label{subsec:background-properties-probabilistic}

In our terminology, an MPC language is \textbf{probabilistic} if it supports a primitive for drawing uniform, random samples from a
discrete, finite set. For example, drawing a uniform, random sample from ⸨𝔹⸩, the booleans, corresponds to a fair coin toss. We
could also draw from ⸨ℕ⸤32⸥⸩, natural numbers modulo ⸨2^{32}⸩. Why is such a primitive important in an MPC language?

As discussed in Chapter~\ref{sec:intro}, MPC languages rely on ORAM to support a RAM-model of computation. In their seminal paper,~\citet{}
showed that Trivial ORAM, which has an overhead of $O(n)$, is optimal if the oblivious scheme is deterministic. However, provided the ability
to perform uniform, random samples, it is possible to design more efficient ORAMs. For example, all tree-based ORAM schemes, including
(Recursive) Circuit ORAM, rely the position tags being uniform, random samples. Additionally, there are highly
optimized oblivious algorithms and data structures which do not rely on ORAM, but do rely on randomness to achieve greater efficiency~\cite{}.

To see how the use of random sampling is used to improve efficiency, consider the \mpc{} code in Figure~\ref{fig:mpc-2-oram}. The purpose of
this code is to allow \bob to choose one of two secrets belonging to \alice without \alice learning which one \bob chose. On lines 1-2,
\alice shares two secrets, ⸨s⸤A1⸥⸩ and ⸨s⸤A2⸥⸩, with \bob. On line 3, \bob's choice, ⸨s⸤B⸥⸩, is shared with \alice.
On line 4, \bob flips a coin, ⸨u⸩, which he shares with \alice. On lines 5-6, the array, ⸨a⸩, is populated
in-order if the coin ⸨u⸩ is heads, and out-of-order if the coin is tails. Finally, on line 7 the XOR of \bob's choice with the flipped
coin is revealed to both parties. Notice that if \bob's choice was ⸨0⸩ (tails) then he always gets ⸨s⸤A1⸥⸩ and
likewise for ⸨1⸩ (heads) and ⸨s⸤A2⸥⸩.

This code illustrates the importance of random sampling for efficient RAM access. If the MPC program had instead stored the array ⸨a⸩ in
a Trivial ORAM then the lookup on line 8 would require 2 accesses (one for each element in the ORAM). This may not seem so bad, but consider
the overhead if \bob wanted to choose from an array of length ⸨n⸩. In that case, the lookup on line 8 would incur an $O(n)$ overhead!

\begin{figure}[h]
M⁅
\begin{array}{r@{␠}lcl}
   «0:» & 𝑚𝑐3l{ ⦑par⦒[\alice,\bob] }
\\ «1:» & ␠⦑let⦒␣s⸤A1⸥ ⧼=⧽ ⦑share⦒[\alice → \alice,\bob]␣(⦑par⦒[\alice]␣⦑read⦒␣ℤ)␣⦑in⦒
\\ «2:» & ␠⦑let⦒␣s⸤A2⸥ ⧼=⧽ ⦑share⦒[\alice → \alice,\bob]␣(⦑par⦒[\alice]␣⦑read⦒␣ℤ)␣⦑in⦒
\\ «3:» & ␠⦑let⦒␣s⸤B⸥  ⧼=⧽ ⦑share⦒[\bob → \alice,\bob]␣⦑par⦒[\bob]␣⦑read⦒␣⦑in⦒
\\ «4:» & ␠⦑let⦒␣u     ⧼=⧽ ⦑share⦒[\bob → \alice,\bob]␣⦑par⦒[\bob]␣𝒰(❴0,1❵)␣⦑in⦒
\\ «5:» & ␠⦑let⦒␣l,r   ⧼=⧽ ⦑mux⦒␣u␣⦑then⦒␣s⸤A1⸥,s⸤A2⸥␣⦑else⦒␣s⸤A2⸥,s⸤A1⸥␣⦑in⦒
\\ «6:» & ␠⦑let⦒␣a     ⧼=⧽ [l;␣r]␣⦑in⦒
\\ «7:» & ␠⦑let⦒␣idx   ⧼=⧽ ⦑reveal⦒[\alice,\bob]␣s⸤B⸥ ⊕ u␣⦑in⦒
\\ «8:» & ␠⦑let⦒␣r     ⧼=⧽ a[idx]␣⦑in⦒
\\ «9:» & ␠…
\end{array}
M⁆
\caption{Conceptual 2 element ORAM lookup in \mpc{}}
\label{fig:mpc-2-oram}
\end{figure}

\subsection{High Assurance MPC Languages}
\label{subsec:background-properties-assurance}

Finally, an MPC language is \textbf{high assurance} if it guarantees that programs are oblivious by construction.
Existing MPC languages with support for uniform, random sampling cannot be considered \textbf{high assurance} because
they do not allow developers to distinguish between \emph{declassification} and \emph{revelation}. A declassification
intends to leak some necessary information about secrets. The information leak is part of the specification.
A revelation intends not to leak any information about secrets. Freedom from information leaks is
part of the specification. For example, the Millionaire's Problem in Figure~\ref{fig:millionaires-symphony} features a declassification. The
developer expects that declassification of the comparison will leak some information about the net worth of the
participants. In contrast, the ``declassifications'' in the One-Time Pad and Mini-ORAM examples (Figures~\ref{fig:otp-symphony} and~\ref{fig:mpc-2-oram})
are really revelations. The developer expects that the ``declassifications'' will not leak any information about \alice's secrets.

Existing MPC languages provide only a mechanism for declassifications. These declassifications are trusted and any
information leaked is assumed to be intentional. Unfortunately, this means that revelations must also use the
declassification mechanism. As such, the leak-freedom of those revelations cannot be checked
by the language. For example, consider what happens if we change line 7 in the Mini-ORAM example to declassify ⸨s⸤B⸥⸩.
In that case, the functional correctness of the program is unaffected but we have accidentally leaked \bob's secret to \alice!
This information leak could have been prevented by a language which provides a revelation mechanism.

Both MPC applications and libraries rely on revelation. Cryptographers rely on revelation to implement secure protocols such as ORAM,
Function Secret Sharing~\cite{}, and Zero-Knowledge Proofs\cite{}. Cryptographers are not the only ones who we expect to leverage the
revelation mechanism. It turns out that many application-level algorithms and data structures also rely on revelation. For example,
various comparison-based algorithms in MPC are optimized by permuting the collection prior to sorting~\cite{hamada2012}. This allows occurences of
comparison in the sorting algorithm to use ⸨⦑if⦒⸩ over revealed comparison, instead of ⸨⦑mux⦒⸩. These sorts of algorithms and
data structures are more likely to be implemented by developers who are not experts in cryptography.

A high assurance language is beneficial to both MPC experts and novices. Experts save time and gain confidence in their
experimental protocols by relying on the language to enforce confidentiality. For example, modifying line 7 in
Figure~\ref{fig:mpc-2-oram} to use a revelation means it need not be manually audited. Instead, the language would ensure that the
revelation is safe. Likewise, MPC novices will benefit by being able to safely experiment with their own cryptographic optimizations
without fear of accidentally leaking information they didn't intend to. They need only mark revelations that are expected to be
leak-free appropriately.

In Chapters~\ref{ch:lam-obliv} and~\ref{ch:proposal} we will formally define a \textbf{high assurance} language as one which provides
a mechanism for revelations and provably correct enforcement mechanism. For all the languages in this proposal, we will
assume that the enforcement mechanism is a static type system. If the type system accepts the program, then all the revelations
are guaranteed to be leak-free. This is proved through an appropriate obliviousness property, PMTO for \obliv and PMTO\% for \lang.

\ins{TODO: refer to PMTO and PMTO\% as theorems}
\ins{IDEA: In future paper, implement quicksort (1) on top of ORAM and then (2) using permutation optimization. Show that (2) is much
  faster, which suggests that building ORAM into the language and using it everywhere is not a good solution.}
