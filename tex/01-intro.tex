\chapter{Introduction}
\label{ch:intro}

\epigraph{It seems like everywhere I go \\
          The more I see, the less I know}{
            Michael Franti \& Spearhead \\
            \emph{Say Hey I Love You}}

Programming languages have a duty of \emph{confidentiality}. They are responsible for protecting the
data that we designate as sensitive or private. Most modern computation is \emph{mutual} --- computation
is performed over the private information of many parties --- which means that programming languages
have a duty of \emph{mutual confidentiality}. What do languages that guarantee mutual confidentiality look
like? How can they be improved?

\section{Secure Multiparty Computation and Obliviousness}
\label{sec:intro}

% What is MPC, in a nutshell?
\emph{Secure Multiparty Computation} (MPC) is a subfield of cryptography
that allows mutually untrusting parties to compute arbitrary functions over their private inputs while revealing nothing
except the function output. That is, MPC allows parties to work together to run programs \emph{under encryption}.

% Problem: MPC programs are hard to write and read for novices.
% Solution: Abstract Sequentiality.
Approaches to MPC have improved significantly over the years. The first full implementation, FairPlay~\cite{todo}, could evaluate
only a few hundred Boolean gates per second. Modern implementations with custom deployments evaluate \emph{billions} of Boolean gates per
second~\cite{todo}, and cryptographers continue to reduce the cost of MPC. Despite increasing efficiency and compelling applications
(secure auctions, secure databases, collaborative machine learning, and any other imaginable application with security concerns),
MPC has not been widely adopted. One significant barrier to adoption is a lack of appropriate infrastructure. Today, it is difficult
for non-experts to understand and work in the complex distributed model that MPC requires. To assuage this difficulty, some recent
MPC lanuages (e.g. SCVM, Wysteria, and Symphony) are \textbf{abstractly sequential}. An \textbf{abstractly sequential} language hides
the parallel deployment of the MPC program from the programmer. This proposal builds heavily on one such language, Symphony, which is
discussed formally in~\cref{sec:lam-mpc}.

% Problem: MPC programs are slow, linear ORAM is intractable.
% Solution: Uniform, random sampling => logarithmic ORAM.

In addition to a lack of infrastructure, MPC languages also fail to provide adequate support for RAM-model programming.
The standard means by which RAM-model computation is supported in MPC languages is using \emph{Oblivious RAM} (ORAM). To date, most MPC languages
simply do not provide any access to ORAM. In their 2019 SoK,~\citet{todo} observe that only a ``few frameworks have ORAM support, either
natively (ObliVM and SCALE-MAMBA) or via a library (Obliv-C).'' The three languages mentioned, ObliVM~\cite{todo}, SCALE-MAMBA~\cite{todo}, and
Obliv-C~\cite{todo}, are \textbf{probabilistic} MPC languages. According to~\citet{todo}, a language must have support for uniform, random sampling
to implement asymptotically optimal ORAM. Indeed, each of these languages provides a highly optimized implementation of
Circuit ORAM~\cite{todo}, which is known to be asymptotically optimal (i.e. $O(\log{n})$).

% Problem: MPC programs that use uniform, random sampling for efficiency are not guaranteed to be safe (confidential).
% Solution: High assurance programs via an oblivious or leak-free declassification which is checked for confidentiality.
It is common practice in probabilistic MPC languages to declassify values with the expectation that they
do not reveal any information. For example, consider the One-Time Pad (OTP) in~\cref{fig:otp-symphony}.

\begin{figure}[h]
M⁅
\begin{array}{r@{␠}lcl}
     «0:» & 𝑚𝑐3l{ ⦑par⦒[\alice,\bob] }
  \\ «1:» & ␠⦑let⦒␣k ⧼=⧽ ⦑par⦒[\alice]␣𝒰(❴0,1❵)␣⦑in⦒
  \\ «2:» & ␠⦑let⦒␣s ⧼=⧽ ⦑par⦒[\alice]␣⦑read⦒␣⦑in⦒
  \\ «3:» & ␠⦑let⦒␣e ⧼=⧽ ⦑declassify⦒[\alice → \alice,\bob]␣⦑par⦒[\alice]␣k ⊕ s␣⦑in⦒
  \\ «4:» & ␠…
\end{array}
M⁆
\caption{\mpc{} code for a One-Time Pad}
\label{fig:otp-symphony}
\end{figure}

This example is a computation between two parties, \alice and \bob. On line 1, \alice generates a local uniform, random bit which will serve
as her encryption key. On line 2, she reads a secret bit from local storage. Finally, on line 3, she encrypts her secret, ⸨s⸩, by XORing it
with the key ⸨k⸩ and then declassifies the result to \bob. Does \bob learn any information about \alice's secret? He doesn't, because it has been encrypted
by the random bit ⸨k⸩ which he doesn't know. As long as ⸨k⸩ is never revealed to \bob, \alice's secret is safe.
However, in existing MPC languages, there is no way to specify that the declassification on line 3 reveals no information about ⸨s⸩.
A \textbf{high assurance} language ensures that programs are probabilistically oblivious. They give programmers the ability to specify
that probabilistic values may be safely revealed to other parties. They gives novices the opportunity to experiment safely, and free experts from having to
manually audit their programs.

In summary, we would like a modern MPC language to satisfy the following properties.

\begin{enumerate}
\item \label{itm:sequential} \textbf{\underline{Abstractly Sequential}} -- An abstractly sequential language, which hides the parallel deployment from the
  programmer, is necessary for lowering the barrier of entry to MPC languages. Being able to program as though the program is being executed sequentially,
  without worrying about the parallel deployment of the program, significantly reduces the complexity of MPC programming.
\item \label{itm:probabilistic} \textbf{\underline{Probabilistic}} -- A probabilistic language, which allows sampling from uniform, random distributions,
  is necessary for implementing asymptotically optimal ORAM, oblivious algorithms, and oblivious data structures. When more efficient oblivious protocols are
  invented (which happens every year), cryptographers would like to be able to implement these protocols as libraries in MPC languages.
\item \label{itm:assurance} \textbf{\underline{High Assurance}} -- A high assurance language, which guarantees that all programs are oblivious, is necessary
  for certifying that programs do not leak information through side channels. Without certification of obliviousness, a domain expert
  would need to manually audit the declassifications in the program which are expected to reveal no information. Such declassifications
  are common in, for example, tree-based ORAM constructions.
\end{enumerate}

\paragraph{Problem.} No existing MPC language is (all together) \textbf{Abstractly Sequential}, \textbf{Probabilistic}, and \textbf{High Assurance}.

\section{Proposed Work}
\label{sec:intro-proposal}

\paragraph{Hypothesis.} It is possible to design and implement an efficient language for MPC which is \textbf{Abstractly Sequential},
\textbf{Probabilistic}, and \textbf{High Assurance} (as defined above).

\paragraph{Contribution: \obliv is PMTO.} In prior work, we show that ObliVM's~\cite{todo} type system is not oblivious,
and show how to fix it in such a way that is satisfies \emph{Probabilistic Memory Trace Obliviousness} (PMTO).
We prove this property for \obliv, a non-MPC language for oblivious programming.

\paragraph{Contribution: \mpc is MTO\%.} In ongoing work, we show that \mpc's type system guarantees that MPC programs are
\emph{Memory Trace Oblivious Modulo Declassifications} (MTO\%). To our knowledge, this is the first time such a theorem has been
proved for an MPC language. The SCVM language has a proof of a related property based on crypto-style ``simulation.''

\paragraph{Contribution: Identifying PMTO\%.} We identify and define \emph{Probabilistic Memory Trace Oblivious Modulo Declassification}
(PMTO\%) as an appropriate definition of security for probabilistic MPC languages. We also investigate and explicate the connection between
this property and other common security properties involving declassifications (such as Gradual Release~\cite{todo}).

The contributions described above have already been accomplished. They are necessary but not sufficient to confirm our hypothesis.
To confirm our hypothesis, we propose the design and implementation of a new language for MPC, called \lang, which extends the \mpc
language with primitives for drawing uniform, random samples and declassifying them.

\paragraph{Contribution: \lang, an \textbf{Abstractly Sequential}, \textbf{Probabilistic}, \textbf{High Assurance} MPC language.}
\lang will extend \mpc with expressions for uniform, random sampling and declassification. Having done so, we will satisfy
Property~\ref{itm:probabilistic} above. Next, we will prove that \lang satisfies type safety, simulation, and obliviousness theorems.
The type safety and simulation theorems will establish that \lang satisfies Property~\ref{itm:sequential}. The type safety and obliviousness
theorems will establish that \lang satisfies Property~\ref{itm:assurance}. We will provide evidence that \lang is expressive and efficient
by implementing modern ORAM protocols (Trivial ORAM, Circuit ORAM, and Recursive (Circuit) ORAM), and empirically evaluating them to ensure
that they have the appropriate asymptotic behavior. Having accomplished all this, we will have confirmed our hypothesis.

\Cref{tab:timeline} describes the remaining tasks and gives a rough estimate for their time to completion. Tasks
in \colorbox{implColor}{red} require adding features to the existing Haskell interpreter for \lang. Tasks in \colorbox{theoryColor}{blue}
require adaptation of the formal metatheory of \mpc.

\begin{table*}
  \centering
  \small

\begin{tabular}{|p{.33\textwidth}|p{.33\textwidth}|p{.33\textwidth}|}
  \hline
  \textbf{Task} & \textbf{Description} & \textbf{Estimated Time of Completion} \\
  \hline

  \rowcolor{implColor}
  Probabilistic Choice &
  Implement probabilistic choice and declassification for \lang. &
  \ins{TODO} \\ \hline

  \rowcolor{implColor}
  Type Checker &
  Implement the type checker for \lang. &
  \ins{TODO} \\ \hline

  \rowcolor{implColor}
  Case Studies:
  \begin{itemize}
  \item Trivial ORAM
  \item Circuit ORAM
  \item Recursive (Circuit) ORAM
  \end{itemize} &
  Implement the ORAM case studies in \lang and benchmark their performance. &
  \ins{TODO} \\ \hline

  \rowcolor{theoryColor}
  Static Semantics &
  Design the type system of \lang. &
  \ins{TODO} \\ \hline

  \rowcolor{theoryColor}
  Type Soundness Proof &
  Prove \lang satisfies Type Soundness. &
  \ins{TODO} \\ \hline

  \rowcolor{theoryColor}
  Simulation Proof &
  Prove \lang satisfies Simulation. &
  \ins{TODO} \\ \hline

  \rowcolor{theoryColor}
  PMTO\% Proof &
  Prove \lang satisfies PMTO\%. &
  \ins{TODO} \\ \hline

  Thesis Writing &
  Write and defend the dissertation. &
  \ins{TODO} \\ \hline
\end{tabular}
  \caption{
    Description of remaining tasks and an estimate of their time to completion.
    }
\label{tab:timeline}
\end{table*}

\chapter{Background}
\label{ch:background}

\epigraph{Trust in me in all you do \\
          Have the faith I have in you \\
          Love will see us through \\
          If only you trust in me}{
            Etta James \\
            \emph{Trust in Me}}

The purpose of this chapter is to answer one question. Why should we care about the properties outlined
in~\Cref{sec:background-properties}? To answer this question, we'll first need to describe the \mpc language
in more depth so that it may be used as the \textit{lingua franca} for the remainder of the proposal.
We will assume that readers are familiar with MPC and ORAM. For readers who don't have this
background,~\Cref{ch:gmw,ch:oram} provide an overview of the GMW MPC protocol and ORAM respectively.

\section{\mpc By Example}
\label{sec:background-symphony}

As a means of gentle introduction, let's consider the classic Millionaire's Problem in which two wealthy parties,
\alice and \bob, would like to know who is wealthier.

\begin{figure}[h]
M⁅
\begin{array}{r@{␠}lcl}
   «0:» & 𝑚𝑐3l{ ⦑par⦒[\alice,\bob] }
\\ «1:» & ␠⦑let⦒␣x    ⧼=⧽ ⦑par⦒[\alice]␣⦑read⦒␣ℕ␣⦑in⦒
\\ «2:» & ␠⦑let⦒␣y    ⧼=⧽ ⦑par⦒[\bob]␣⦑read⦒␣ℕ␣⦑in⦒
\\ «3:» & ␠⦑let⦒␣sx   ⧼=⧽ ⦑share⦒[\alice → \alice,\bob]␣x␣⦑in⦒
\\ «4:» & ␠⦑let⦒␣sy   ⧼=⧽ ⦑share⦒[\bob → \alice,\bob]␣y␣⦑in⦒
\\ «5:» & ␠⦑let⦒␣r    ⧼=⧽ sx < sy␣⦑in⦒
\\ «6:» & ␠⦑let⦒␣z    ⧼=⧽ ⦑reveal⦒[\alice,\bob]␣r␣⦑in⦒
\\ «7:» & ␠…
\end{array}
M⁆
\caption{\mpc{} code for the Millionaire's Problem}
\label{fig:millionaires-symphony}
\end{figure}

The code in~\Cref{fig:millionaires-symphony} is written in the \mpc language. On line 0, the ⸨⦑par⦒[\alice,\bob]⸩ expression
indicates that both \alice and \bob should execute the body of the expression. Any other parties will ignore the
body of the expression and generate an \emph{opaque value}, denoted \opaque, instead. On line 1, \alice ⸨⦑read⦒⸩s her own wealth,
⸨\alices{⋖wealth⋗}⸩, from local storage and binds it to ⸨x⸩. The ⸨ℕ⸩ after ⸨⦑read⦒⸩ indicates that we expect the wealth to be a
natural number. \bob evaluates the expression ⸨⦑par⦒[\alice]␣⦑read⦒␣ℕ⸩ to
\opaque because he is not included in the ⸨⦑par⦒[\alice]⸩ expression. On line 2, \bob reads his wealth, ⸨\bobs{⋖wealth⋗}⸩,
and binds it to ⸨y⸩ while \alice binds \opaque to ⸨y⸩. In summary, after line 2, the local environemnts are:

M⁅
  Aːcc
  A⁅ \alice & \bob
  A⁃ ⟨x ↦ \alices{⋖wealth⋗},␣y ↦ \opaque⟩␠ & ␠⟨x ↦ \opaque,␣y ↦ \bobs{⋖wealth⋗}⟩
  A⁆
M⁆

On line 3, \alice splits her ⸨\alices{⋖wealth⋗}⸩ into two shares. She keeps her share and sends \bob's share to him. We use the same
notation for shares which appears in \Cref{ch:gmw} to emphasize the connection between the explanation of GMW and the execution model
of \mpc. On line 4, \bob splits his ⸨\bobs{⋖wealth⋗}⸩ into two shares. He keeps his share and send's \alice's share to her. So, after
line 4 the local environments are:

M⁅
  Aːllll
  A⁅ 𝑚𝑐2c{\alice} & 𝑚𝑐2c{\bob}
    A⁃ ⟨…,&␣sx ↦ \aliceSh{\alices{⋖wealth⋗}}, & ␠⟨…,&␣sx ↦ \bobSh{\alices{⋖wealth⋗}},
    A⁃    &␣sy ↦ \aliceSh{\bobs{⋖wealth⋗}}⟩   &     &␣sy ↦ \bobSh{\bobs{⋖wealth⋗}}⟩
  A⁆
M⁆

On line 5, \alice and \bob compute over their shares to produce a share indicating who is wealthier. Here we assume that the language knows
how to compute ⸨<⸩ over shares. For example, if the underlying MPC protocol were GMW then we would encode ⸨<⸩ as a magnitude circuit made
up of XOR and AND gates.

M⁅
  Aːcc
  A⁅ \alice & \bob
  A⁃ ⟨…,␣r ↦ \aliceSh{A⸨\alices{⋖wealth⋗} < \bobs{⋖wealth⋗}A⸩}⟩ & ␠⟨…,␣r ↦ \bobSh{A⸨\alices{⋖wealth⋗} < \bobs{⋖wealth⋗}A⸩}⟩
  A⁆
M⁆

Finally, on line 6, the shares of ⸨\alices{⋖wealth⋗} < \bobs{⋖wealth⋗}⸩ among \alice and \bob are combined and the
result is revealed to both parties. The XOR operator, ⸨⊕⸩, is used to combine shares implicitly as part of the ⸨⦑reveal⦒⸩.

M⁅
  Aːllll
  A⁅ 𝑚𝑐2c{\alice} & 𝑚𝑐2c{\bob}
    A⁃ ⟨…,␣z &{} ↦ \aliceSh{A⸨\alices{⋖wealth⋗} < \bobs{⋖wealth⋗}A⸩} & ␠⟨…,␣z &{} ↦ \aliceSh{A⸨\alices{⋖wealth⋗} < \bobs{⋖wealth⋗}A⸩}
    A⁃       &{}␣⊕ \bobSh{A⸨\alices{⋖wealth⋗} < \bobs{⋖wealth⋗}A⸩} & &{}␣⊕ \bobSh{A⸨\alices{⋖wealth⋗} < \bobs{⋖wealth⋗}A⸩}
    A⁃       &{}␣= \alices{⋖wealth⋗} < \bobs{⋖wealth⋗}⟩ & &{}␣= \alices{⋖wealth⋗} < \bobs{⋖wealth⋗}⟩
  A⁆
M⁆

The Millionaire's Problem in~\Cref{fig:millionaires-symphony} illustrates two exotic features of \mpc. First, the programmer
controls which parties evaluate which expressions by using ⸨⦑par⦒⸩ expressions. This is in contrast to languages like OblivC
which must coordinate parties manually using ⸨⦑if⦒⸩ expressions. Second, plain and encrypted computations may be arbitrarily
interleaved. There is no phase distinction between plaintext and encrypted computation. Languages such as OblivC and Wysteria
require that MPC computation be separated from plaintext computation syntactically.

\section{Desirable Properties of MPC}
\label{sec:background-properties}

In~\Cref{sec:intro} we claimed that a MPC language ought to be \textbf{Abstractly Sequential}, \textbf{Probabilistic},
and \textbf{High Assurance}. But, what are these properties and why do we care about them?

\subsection{Abstractly Sequential}
\label{subsec:background-properties-centralized}

A language which is abstractly sequential liberates the programmer from managing a distributed computation. In the context of MPC,
the programmer does not need to perform commands conditionally on which party is executing. To see the difference, contrast the
Obliv-C implementation of the Millionaire's Problem in~\Cref{fig:millionaires-oblivc} with the \mpc implementation
in~\Cref{fig:millionaires-symphony}.

\begin{figure}[h]
\begin{lstlisting}[language=c,basicstyle=\footnotesize\ttfamily,numbers=left,stepnumber=1]
  // File: million.h
  typedef struct {
    int in;
    bool out;
  } ProtocolIO;

  void millionaire (void *args);

  // File: million.oc
  #include <million.h>
  #include <obliv.oh>

  void millionaire (void *args) {
    ProtocolIO *io = args;
    obliv int a, b;
    obliv bool res = false;
    a = feedOblivInt (io->in, 1);
    b = feedOblivInt (io->in, 2);
    obliv if (a < b) res = true;
    revealOblivBool(&io->out, res, 0);
  }

  // File: million.c
  #include <million.h>

  int main (int argc, char *argv[]) {
    ProtocolDesc pd;
    ProtocolIO io;
    int p = (argv[1] == '1' ? 1 : 2);
    sscanf(argv[2], "%d", &io.in);
    // ... set up TCP connections

    setCurrentParty(&pd, p);
    execYaoProtocol(&pd, millionaire, &io);
    printf ("%d\n", io.out);
    // ... cleanup
  }
\end{lstlisting}
\caption{Obliv-C code for the Millionaire's Problem}
\label{fig:millionaires-oblivc}
\end{figure}

Hopefully it is clear that many of the details of the distributed deployment have leaked into the application code.
The \lstinline[language=c,basicstyle=\ttfamily]{main} function is responsible for scaffolding such as establishing TCP connections (line 31),
explicitly executing the protocol (line 34), and populating the \lstinline[language=c,basicstyle=\ttfamily]{io} variable with party inputs (line 30). As
a point of comparison, take a look back at~\Cref{fig:millionaires-symphony}. Notice that this code makes no mention of TCP connections,
explicit execution of any protocol, or eagerly reading and caching secret inputs. Dealing with such infrastructure can be annoying, but
is it safe? By allowing the programmer to execute code conditionally based on which party is executing, languages like Obliv-C allow code
which has all the usual pitfalls of distributed programming. For example, deadlock caused by both parties attempting to receive values
from each other before they have performed a send.

In Chapter~\ref{ch:lam-mpc} we formally define \textbf{Abstractly Sequential} MPC languages as those which satisfy~\Cref{thm:mpc-safety,thm:mpc-simulation}.
~\Cref{thm:mpc-safety} guarantees that well-typed \mpc programs don't get stuck (encounter a runtime error).~\Cref{thm:mpc-simulation} guarantees
that the sequential interpretation of a program terminates successfully if and only if the distributed interpretation terminates successfully (with the same
result). These two theorems work together to guarantee that well-typed programs can't encounter liveness errors.

\subsection{Probabilistic}
\label{subsec:background-properties-probabilistic}

In our terminology, an MPC language is \textbf{Probabilistic} if it supports a primitive for drawing uniform, random samples from a
discrete, finite set. For example, drawing a uniform, random sample from ⸨𝔹⸩, the Booleans, corresponds to a fair coin toss. We
could also draw from ⸨ℕ⸤32⸥⸩, Natural numbers modulo ⸨2^{32}⸩. Why is such a primitive important in an MPC language?

As discussed in Chapter~\ref{sec:intro}, MPC languages rely on ORAM to support a RAM-model of computation. In their seminal paper,~\citet{}
showed that Trivial ORAM, which has an overhead of $O(n)$, is optimal in a deterministic language. However, provided the ability
to perform uniform, random samples, it is possible to design more efficient ORAMs. For example, tree-based ORAM schemes, including
(Recursive) Circuit ORAM, require the position tag to be uniformly random. In addition to ORAMs, there are optimized oblivious algorithms
and data structures which do not rely on ORAM, but do rely on randomness to achieve greater efficiency~\cite{}.

To see how the use of random sampling is used to improve efficiency, consider the \mpc{} code in~\Cref{fig:mpc-2-oram}. The purpose of
this code is to allow \bob to choose one of two secrets belonging to \alice without \alice learning which one \bob chose. On lines 1-2,
\alice shares two secrets, ⸨s⸤A1⸥⸩ and ⸨s⸤A2⸥⸩, with \bob. On line 3, \bob's choice, ⸨s⸤B⸥⸩, is shared with \alice.
On line 4, \bob flips a coin, ⸨u⸩, which he shares with \alice. On lines 5-6, the array, ⸨a⸩, is populated
in-order if the coin ⸨u⸩ is heads (⸨0⸩), and out-of-order if the coin is tails (⸨1⸩). Finally, on line 7 the XOR of \bob's choice with the
coin is revealed to both parties. Notice that if \bob's choice was ⸨0⸩ (heads) then he always gets ⸨s⸤A1⸥⸩ and
likewise for ⸨1⸩ (tails) and ⸨s⸤A2⸥⸩.

This code illustrates the importance of random sampling for efficient RAM access. If the MPC program had instead stored the array ⸨a⸩ in
a Trivial ORAM then the lookup on line 8 would require 2 accesses (one for each element in the ORAM). This may not seem so bad, but consider
the overhead if \bob wanted to choose from an array of length ⸨n⸩. In that case, the lookup would incur an $O(n)$ overhead!

\begin{figure}[h]
M⁅
\begin{array}{r@{␠}lcl}
   «0:» & 𝑚𝑐3l{ ⦑par⦒[\alice,\bob] }
\\ «1:» & ␠⦑let⦒␣s⸤A1⸥ ⧼=⧽ ⦑share⦒[\alice → \alice,\bob]␣(⦑par⦒[\alice]␣⦑read⦒␣ℤ)␣⦑in⦒
\\ «2:» & ␠⦑let⦒␣s⸤A2⸥ ⧼=⧽ ⦑share⦒[\alice → \alice,\bob]␣(⦑par⦒[\alice]␣⦑read⦒␣ℤ)␣⦑in⦒
\\ «3:» & ␠⦑let⦒␣s⸤B⸥  ⧼=⧽ ⦑share⦒[\bob → \alice,\bob]␣(⦑par⦒[\bob]␣⦑read⦒␣ℕ⸤1⸥)␣⦑in⦒
\\ «4:» & ␠⦑let⦒␣u     ⧼=⧽ ⦑share⦒[\bob → \alice,\bob]␣(⦑par⦒[\bob]␣𝒰(ℕ⸤1⸥))␣⦑in⦒
\\ «5:» & ␠⦑let⦒␣⟨l,r⟩ ⧼=⧽ u == 0 ¿ ⟨s⸤A1⸥,s⸤A2⸥⟩ ◇ ⟨s⸤A2⸥,s⸤A1⸥⟩␣⦑in⦒
\\ «6:» & ␠⦑let⦒␣a     ⧼=⧽ [l;␣r]␣⦑in⦒
\\ «7:» & ␠⦑let⦒␣idx   ⧼=⧽ ⦑reveal⦒[\alice,\bob]␣s⸤B⸥ ⊕ u␣⦑in⦒
\\ «8:» & ␠⦑let⦒␣r     ⧼=⧽ a[idx]␣⦑in⦒
\\ «9:» & ␠…
\end{array}
M⁆
\caption{Conceptual 2 element ORAM lookup in \mpc{}}
\label{fig:mpc-2-oram}
\end{figure}

\subsection{High Assurance}
\label{subsec:background-properties-assurance}

An MPC language is \textbf{High Assurance} if it guarantees that programs are oblivious by construction.
Existing MPC languages with support for uniform, random sampling cannot be considered high assurance because
they do not allow developers to distinguish between \emph{declassification} and \emph{benign exposure}.
A declassification intends to leak some necessary information about secrets. The information leak is part of
the specification of the program. A benign exposure intends not to leak any information about secrets. Freedom
from information leaks is part of the specification. For example, the Millionaire's Problem in~\Cref{fig:millionaires-symphony}
features a declassification. The developer expects that declassification of the comparison will leak some information about the wealth of the
participants. In contrast, the ⸨⦑reveal⦒⸩ expressions in the One-Time Pad and Mini-ORAM examples (~\Cref{fig:otp-symphony,fig:mpc-2-oram})
are benign exposures. The developer expects that the ⸨⦑reveal⦒⸩ is not actually leaking any information about the secrets
of either party.

Existing MPC languages provide only a mechanism for declassification. These declassifications are trusted and any
information leaked is assumed to be intentional. Unfortunately, this means that benign exposures must also use the
declassification mechanism. As such, the leak-freedom of those exposures is not, and cannot be, checked by the language.
Consider what happens if we change line 7 in the Mini-ORAM example to declassify ⸨s⸤B⸥⸩.
In that case, the functional correctness of the program is unaffected but we have accidentally leaked \bob's secret to \alice!
This information leak could have been prevented by a language which provides a mechanism to certify that benign exposures do
not leak any information.

Both MPC applications and libraries rely on benign exposure. Cryptographers rely on benign exposure to implement secure protocols such as ORAM,
Function Secret Sharing~\cite{}, and Zero-Knowledge Proofs\cite{}. Cryptographers are not the only ones who we expect to leverage the
benign exposure mechanism. It turns out that many application-level algorithms and data structures also rely on benign exposure. For example,
various comparison-based algorithms in MPC are optimized by permuting the collection prior to sorting~\cite{hamada2012}. This allows occurrences of
comparison in the sorting algorithm to use ⸨⦑if⦒⸩ over declassified comparison, instead of a multiplexor. These sorts of algorithms and
data structures are more likely to be implemented by developers who are not experts in cryptography.

A high assurance language is beneficial to both MPC experts and novices. Experts save time and gain confidence in their
experimental protocols by relying on the language to enforce confidentiality. For example, modifying line 7 in
Figure~\ref{fig:mpc-2-oram} to use a benign exposure means it need not be manually audited. Instead, the language would ensure that the
exposure is safe. Likewise, MPC novices will benefit by being able to safely experiment with their own cryptographic optimizations
without fear of accidentally leaking information they didn't intend to. They need only mark exposures that are expected to be
leak-free appropriately.

In~\Cref{ch:lam-obliv,ch:proposal} we formally define \textbf{High Assurance} languages as those which satisfy~\Cref{thm:mpc-pmto} and~\Cref{thm:lang-pmto}
respectively. In the former, we discuss \obliv which is not an MPC language and doesn't have declassifications. It only contains what we referred to above
as benign exposure. In the latter, we discuss \lang which is an MPC language and therefore has declassifications. As such, we must generalize~\Cref{thm:mpc-pmto}
to~\Cref{thm:lang-pmto} to take declassifications into account. In both languages, a sound static type system is used as the enforcement mechanism. Only well-typed
programs are guaranteed to be oblivious.

\ins{IDEA: In future paper, implement quicksort (1) on top of ORAM and then (2) using permutation optimization. Show that (2) is much
  faster, which suggests that building ORAM into the language and using it everywhere is not a good solution.}
