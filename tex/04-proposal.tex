\chapter{Proposal: \lang, A Secure MPC Language With User-Defined ORAM}
\label{ch:proposal}

\ins{Just describe the language as best you can, and then imagine doing the actual work and anticipate, write down issues.
  For example, when describing re-establishing simulation, describe issue w/ making sequential interpretation probabilistic.
  Should we allow uniform, random stuff which is not secret shared? Arguments for yes: labels are a combination of locatedness
  and protocol. Arguments for no: involves communicating between parties w/o secret sharing, also may not be useful.
  How do we establish PMTO\% when we are allowed to cast uniform values to public? It was important in lam-obliv that when you
  casted to secret that implied the value wouldn't be revealed.}



We will construct the \lang language by adding two features from the \obliv language to \mpc. In particular, \mpc lacks
the ability to draw uniform, random samples and to mark a revelation (declassification) as oblivious.

Having seen \mpc and \obliv formally, we are now in a position to explain how the MPC language, \lang, will be constructed.
We will start by making \lang \textbf{probabilistic} by extending \mpc with a feature for uniform, random sampling. This new
feature will be analogous to the ⸨⦑unif⦒⸩ feature of \obliv. By extending \mpc we expect that the language will already be ``mostly''
\textbf{abstractly centralized}. However, we will need to establish this formally by proving that \lang satisfies an appropriate
version of the~\nameref{thm:mpc-simulation} theorem from~\cref{sec:lam-mpc-simulation}. Finally, \lang will be made
\textbf{high assurance} by extending \lang with a feature for zero-information declassification of random values. This feature
is analogous to the ⸨⦑cast⦒⸩ feature of \obliv. We intend for the type system to ensure that uses of ⸨⦑cast⦒⸩ are appropriate
--- i.e. that they really don't reveal any information to other parties. Achieving this will require updating the type system
of \lang with the affinity and probability region features of the \obliv language. Again, we will need to establish that the
type system achieves its goal by proving~\nameref{thm:lang-pmto}.

As a means of confirming that \lang is useful, we will build on the existing Haskell implementation of \mpc. That implementation
includes only a centralized interpreter which does not actually perform any MPC. Adding on to this centralized interpreter, we will
add a distributed interpreter which uses the EMP MPC framework as the MPC backend. Finally, we will implement a type checker based
on the formal type system. Having done this, we will implement a number of case studies in \lang. At a minimum, we will implement
the ORAM schemes described in~\cref{ch:oram} --- Trivial ORAM, Circuit ORAM, and Recursive (Circuit) ORAM. Having done so, we will
have confirmed our type system is expressive enough to type check state of the art ORAM protocols. Finally, using our distributed
interpreter we will confirm that the ORAM schemes have the expected asymptotic complexity (and should therefore be competitive with
the ORAM protocols in existing languages like Obliv-C and SCALE-MAMBA).

\section{Design}

The syntax of \lang is shown in~\cref{fig:lang-syntax}. The security label lattice is the same as
the \mpc language. Labels are a combination of location and protocol. As such, we no longer need the ⸨‹P›⸩ and ⸨‹S›⸩ labels
of \obliv. \lang takes the uniform type of \obliv and adds a protocol to it, ⸨⦑½⦒ μ⸢ρ ; ψ⸣⸩, indicating that random values
can be shared. The atomic conditional works as described in \obliv, returning a tuple of its arguments in-order if the guard
is true, and permuted otherwise. As in \obliv, this is necessary in an affine type system. Tuple elimination is now by
pattern-matching instead of projection, again for compatibility with affinity. The only non-trivial change is the additon of
uniform sampling, ⸨⦑unif⦒⸢ρ⸣␣μ⸩, and casting, ⸨⦑cast⦒⸤ψ⸥[p]␣x⸩.

\ins{observation: pmto is hard when you are allowed to declassify b/c casting to secret doesn't imply that value will never be revealed.} \\

\ins{The goal is to sufficiently explain~\nameref{thm:lang-simulation} and~\nameref{thm:lang-pmto}.}

\begin{theorem}[Forward Simulation] \label{thm:lang-simulation}
  If ⸨ς —→⋆ ⇡~{ς′}⸩, ⸨⇡~{ς′}⸩ is terminal, ⸨ς↯ ↝⋆ ⇡~{G}⸩ and ⸨⇡~{G} ⫽↝⸩, then ⸨⇡~{G} = ⇡~{ς′}↯⸩.
\end{theorem}

\begin{theorem}[PMTO\%] \label{thm:lang-pmto}
  If ⸨ς₁ : τ⸩, ⸨ς₂ : τ⸩, ⸨ς₁ ≈⸤l⸥ ς₂⸩, ⸨ς₁ ⇢⋆ ⇡~{t₁}⸩, ⸨ς₂ ⇢⋆ ⇡~{t₂}⸩, and ⸨R⸤l⸥(⇡~{t₁}) = R⸤l⸥(⇡~{t₂})⸩, then ⸨⇡~{t₁} ⇡~≈⸤l⸥ ⇡~{t₂}⸩.
\end{theorem}

F⁅
\begingroup
\setlength\arraycolsep{0pt} % default is 6pt
\smaller
D⁅
M⁅
Aːrcrcl@{␠}l
A⁅ b     ⧼∈⧽ 𝔹            ⧼ ⧽                                & ⟪booleans⟫
A⁃ i     ⧼∈⧽ ℤ            ⧼ ⧽                                & ⟪integers⟫
A⁃ A,B,C ⧼∈⧽ ‹party›      ⧼ ⧽                                & ⟪parties⟫
A⁃ m,p,q ⧼∈⧽ ‹party-set›  ⧼≜⧽ ℘(‹party›) ⩴ ❴A,…,A❵          & ⟪sets of parties⟫
A⁃ ρ     ⧼∈⧽ R           ⧼ ⧽                                 & ⟪probability region⟫
A⁃ ψ     ⧼∈⧽ ‹prot›       ⧼⩴⧽ ⋅                             & ⟪cleartext⟫
A⁃       ⧼ ⧽              ⧼¦⧽ ⦑enc⦒⋕m                        & ⟪encrypted⟫
A⁃ μ     ⧼∈⧽ ‹base-type›  ⧼⩴⧽ ⦑int⦒                         & ⟪integer type⟫
A⁃       ⧼ ⧽              ⧼¦⧽ ⦑bool⦒                         & ⟪boolean type⟫
A⁃ σ     ⧼∈⧽ ‹loc-type›   ⧼⩴⧽ μ⸢ρ ; ψ⸣                      & ⟪protocol type⟫
A⁃       ⧼ ⧽              ⧼¦⧽ ⦑½⦒ μ⸢ρ ; ψ⸣                   & ⟪uniform type⟫
A⁃       ⧼ ⧽              ⧼¦⧽ τ ᵐ→ τ                         & ⟪function type⟫
A⁃       ⧼ ⧽              ⧼¦⧽ τ × τ                          & ⟪pair type⟫
A⁃ τ     ⧼∈⧽ ‹type›       ⧼⩴⧽ σ@m                           & ⟪located type⟫
A⁃ x,y,z ⧼∈⧽ ‹var›        ⧼ ⧽                                & ⟪variables⟫
A⁃ ⊙     ⧼∈⧽ ‹binop›      ⧼ ⧽                                & ⟪binary operations (e.g., plus, times)⟫
A⁃ e     ⧼∈⧽ ‹expr›       ⧼⩴⧽ x                             & ⟪variable reference⟫
A⁃       ⧼ ⧽              ⧼¦⧽ i                              & ⟪integer literal⟫
A⁃       ⧼ ⧽              ⧼¦⧽ b                              & ⟪boolean literal⟫
A⁃       ⧼ ⧽              ⧼¦⧽ e ⊙ e                          & ⟪binary operation⟫
A⁃       ⧼ ⧽              ⧼¦⧽ e ¿ e ◇ e                      & ⟪atomic conditional⟫
A⁃       ⧼ ⧽              ⧼¦⧽ ⟨e,e⟩                          & ⟪pair creation⟫
A⁃       ⧼ ⧽              ⧼¦⧽ ⦑let⦒␣x,y = e␣⦑in⦒␣e           & ⟪tuple elimination⟫
A⁃       ⧼ ⧽              ⧼¦⧽ ⦑par⦒[p]␣e                     & ⟪parallel execution⟫
A⁃       ⧼ ⧽              ⧼¦⧽ ⦑read⦒␣μ                       & ⟪read int input⟫
A⁃       ⧼ ⧽              ⧼¦⧽ ⦑write⦒␣e                      & ⟪write output⟫
A⁃       ⧼ ⧽              ⧼¦⧽ ⦑embed⦒[p]␣e                   & ⟪encrypted a known constant⟫
A⁃       ⧼ ⧽              ⧼¦⧽ ⦑share⦒[p→p]␣e                 & ⟪share encrypted value⟫
A⁃       ⧼ ⧽              ⧼¦⧽ ⦑declassify⦒[p]␣e              & ⟪declassify encrypted value⟫
A⁃       ⧼ ⧽              ⧼¦⧽ ⦑unif⦒⸢ρ⸣␣μ                    & ⟪uniform, random sample in region⟫
A⁃       ⧼ ⧽              ⧼¦⧽ ⦑cast⦒⸤ψ⸥[p]␣x                 & ⟪cast from uniform⟫
A⁃       ⧼ ⧽              ⧼¦⧽ λ⸤z⸥x⍪ e                       & ⟪(recursive) function creation⟫
A⁃       ⧼ ⧽              ⧼¦⧽ e␣e                            & ⟪function elimination⟫
A⁃       ⧼ ⧽              ⧼¦⧽ ⦑let⦒␣x=e␣⦑in⦒␣e               & ⟪let binding⟫
A⁆
M⁆
D⁆
\endgroup
\caption{\mpc Syntax}
\label{fig:lang-syntax}
F⁆

\section{Implementation}

We must now consider if the formal language we have designed is useful. Can we write interesting programs in the language?
Will those programs be efficient? Questions such as these must be answered empirically. We will address both of these questions
using the ORAM protocols described in~\cref{ch:oram}. In particular, since we are claiming that our type system has a very powerful
property ---~\cref{thm:lang-pmto} --- we might also expect it to be very restrictive. How do we know it actually accepts the ORAM
protocols we would like to prove are oblivious? For this reason, we take ORAM protcols to be representative ``interesting programs.''

The \mpc language has an existing Haskell interpreter. However, this interpreter only implements the centralized semantics of \mpc.
In particular, it doesn't actually perform any MPC and doesn't run as multiple communicating processes. Our first implementation
goal will be to extend the existing interpreter to also support a distribued mode. We have already partially completed this task
by implementing a Haskell FFI to the C++ bindings for EMP. The prototype distributed mode is capable of sharing integers, performing
primitive operations (e.g. addition, comparison), and revealing the result. For example, we have successfully implemented the Millionaire's Problem
from~\ref{sec:background-symphony}. There is more work to support ORAM -- we will need to handle random numbers, references, and
more primitive operations at a minimum. Furthermore, since EMP is limited to two parties, so too is our distributed mode.

The existing \mpc implementation does not have a type checker. We will first add the type checker of \mpc to the implementation without
the additional features (affinity, probability regions) required for \lang. This type system should be sufficient for checking, for example,
that Trivial ORAM\footnote{Interestingly, this will already confirm that Trivial ORAM in the MPC model is~\ref{thm:mpc-mto}.}. Then, we will
add the challenging features of affinity and probability regions. In prior work, we implemented the type system of \obliv and used it to
type check various ORAM case studies. We expect that we will be able to use our experience from that artifact here.

Finally, we can address the two empirical questions mentioned above. First, we will implement the ORAM protocols in our language and confirm
that they type check (and are therefore oblivious). Second, with our distributed mode, we will confirm that the ORAM protocols have the
expected asymptotic behavior. We will perform a series of experiments in which \ins{todo: similar to experiments in FLORAM paper}
This will confirm that our language can support RAM-model computation which is competitive with existing languages that offer ORAM support
(e.g. Obliv-C, SCALE-MAMBA).


\ins{mention somewhere that these ORAM implementations are different from the ones in \obliv -- these are not client-server but instead in
  the ORAM-SC model in which neither party knows the position tag of inserted elements.}
