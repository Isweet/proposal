\chapter{\lang, A Probabilistically Oblivious Language for Secure Multiparty Computation}
\label{ch:proposal}

\mwh{For the parts not yet done, here: Don't formalize these. Instead,
  describe the properties/ideas you imagine you will implement. This
  is research so you don't have to know the result. But you can have
  ideas for the direction, and ideas on the problems you'll have to
  solve. It's worth writing those down.}

Up to this point, we have seen two languages (\mpc and \obliv) which, individually, have all the language
features that we are interested in. \obliv is \textbf{Probabilistic} and \textbf{High Assurance} while \mpc
is designed for MPC and \textbf{Abstractly Sequential}. The purpose of \lang is to unify these two languages
into a single MPC language satisfying all three properties. In this chapter, we formally propose \lang and
provide sketch of how it could be designed and implemented. Along the way, we point out potential difficulties
and hypothesize about how to overcome them.

\section{Design}
\label{sec:proposal-design}

The formal language design for \lang consists of the syntax (\Cref{fig:lang-syntax}), type system,
a probabilistic sequential semantics, and a probabilistic distributed semantics.
The type system should be sound, and additionally it should enforce~\nameref{thm:lang-pmto}. We claim that~\nameref{thm:lang-pmto}
is the appropriate extension of \obliv's~\nameref{thm:obliv-pmto} to the MPC setting. The ``\%" should be read as ``modulo'' and refers
to the PMTO proerty being generalized to handle declassifications. The last theorem we expect \lang to satisfy is~\nameref{thm:lang-simulation}
which is the appropriate extension of \mpc's~\nameref{thm:mpc-simulation} to the probabilistic setting.

\FigureLangSyntax{fig:lang-syntax}{\lang formal syntax.}

\subsection{Syntax}
\label{subsec:proposal-design-syntax}

The proposed syntax of \lang is in~\Cref{fig:lang-syntax}. The syntax of \lang builds primarily
on that of \mpc (\Cref{fig:mpc-syntax}), with the addition of some features from \obliv (\Cref{fig:obliv-syntax}).
The syntax is not presented in ANF (as \mpc was) for clarity. In the future, we will likely transition
to ANF to simplify the metatheory.

Most of \lang's expressions are standard and work identically to \mpc and \obliv.
We have variables ⸨x⸩; integers ⸨i⸩; booleans ⸨b⸩; party sets ⸨p⸩;
binary operations ⸨e ⊙ e⸩;
conditionals ⸨⦑if⦒␣e␣⦑then⦒␣e␣⦑else⦒␣e⸩; multiplexors ⸨e ¿ e ◇ e⸩;
sum creation ⸨ιᵢ␣e⸩ and elimination ⸨⦑case⦒␣x␣❴y⍪e⸤1⸥❵␣❴y⍪e⸤2⸥❵⸩;
pair creation ⸨⟨e,e⟩⸩ and elimination ⸨πᵢ␣e⸩;
reference creation ⸨⦑ref⦒␣e⸩, dereference ⸨¡x⸩, and assignment ⸨x ≔ y⸩;
function creation ⸨λ⸤z⸥x⍪ e⸩ and application ⸨x␣y⸩;
local variable binding ⸨⦑let⦒␣x=e⸤1⸥␣⦑in⦒␣e⸤2⸥⸩;
par(allel) execution ⸨⦑par⦒[e]␣e⸩;
local read ⸨⦑read⦒␣μ⸩ and write ⸨⦑write⦒␣e⸩;
MPC share ⸨⦑share⦒[e→e]␣e⸩ and reveal ⸨⦑reveal⦒⸤b⸥[e]␣e⸩;
uniform distributions ⸨⦑unif⦒⸢ρ⸣␣μ⸩;
and observation of distributions ⸨⦑observe⦒␣e⸩.

Literals no longer have a label ⸨l⸩ attached to them. Instead, we use
the combination of the location annotation ⸨@m⸩ and protocol annotation
⸨ψ⸩ as the information flow label.
The multiplexor, ⸨e₁ ¿ e₂ ◇ e₃⸩, works as in \mpc evaluating to either
⸨v₂⸩ or ⸨v₃⸩. The \obliv multiplexor can be encoded as
⸨e₁ ¿ e₂ ◇ e₃ ⩴ e₁ ¿ ⟨e₂,e₃⟩ ◇ ⟨e₃,e₂⟩⸩. Pairs are eliminated using
projection ⸨πᵢ⸩ as in \mpc rather than with pattern matching ⸨⦑let⦒␣x,y = e␣⦑in⦒␣e⸩
as in \obliv. The local read operation ⸨⦑read⦒␣μ⸩ takes a base type ⸨μ⸩ which is
type we expect to read. The reveal expression ⸨⦑reveal⦒⸤b⸥[e]␣e⸩ takes a boolean
⸨b⸩ indicating if the revelation is expected to be benign (zero-leakage). When
⸨b = ⦑true⦒⸩, we can think of this as acting like ⸨⦑cast⦒⸤‹P›⸥⸩ in \obliv. We rename
the ⸨⦑flip⦒⸢ρ⸣()⸩ expression from \obliv to ⸨⦑unif⦒⸢ρ⸣␣μ⸩ because we may generate
uniform booleans or integers. Finally, the ⸨⦑observe⦒␣e⸩ expression acts like ⸨⦑cast⦒⸤‹S›⸥⸩
from \obliv taking a uniform value and ``observing'' the underlying sample.

Types ⸨τ⸩ have location annotations, just like the values in the sequential semantics of \mpc (\Cref{fig:mpc-seq-aux}).
The located types ⸨σ⸩ refer recursively to types ⸨τ⸩. ⸨⦑prins⦒⸩ is the type of principal sets. The mode annotation ⸨m⸩ on function types ⸨τ ᵐ→ τ⸩ is the static
analog of the mode captured by closures in \mpc. Finally, our base types ⸨μ⸩ are augmented with three annotations. The probability
annotation, ⸨φ⸩, indicates whether this type is uniform or not. ⸨⦑\faThumbsUp⦒⸩ indicates a uniform type (⸨⦑flip⦒⸩ in
\obliv) and ⸨⦑\faThumbsDown⦒⸩ indicates a non-uniform type (⸨⦑bit⦒⸩ in \obliv). The probability region annotation, ⸨ρ⸩,
serves the same purpose as the annotation on the ⸨⦑flip⦒⸩ and ⸨⦑bit⦒⸩ types in \obliv. The protocol annotation, ⸨ψ⸩,
serves the same purpose as the protocol annotation in \mpc. So, for example, the ⸨⦑flip⦒⸢ρ⸣⸩ type from \obliv can be
represented as ⸨⦑bool⦒⸢⦑\faThumbsUp⦒ ; ρ ; ⋅⸣⸩ in \lang. This also allows us to represent, for example, the type
of encrypted uniform distributions of type ⸨μ⸩ among parties ⸨m⸩: ⸨μ⸢⦑\faThumbsUp⦒ ; ρ ; ⦑enc⦒⋕m⸣⸩. We can
also encode types representable in \mpc. For example, ⸨⦑int⦒⸢⦑\faThumbsDown⦒ ; ⊥ ; ⦑enc⦒⋕m⸣⸩, the type of non-random
(⸨⊥⸩) integers encrypted among parties ⸨m⸩.

\subsection{Sequential Semantics}
\label{subsec:proposal-design-seq}

We expect the sequential semantics of \lang to be a standard combination of the sequential
semantics of \mpc (\Cref{fig:mpc-seq}) and \obliv (\Cref{fig:obliv-sem}). In particular, we
expect to be able to take the sequential semantics of \mpc suitably extended and lift them
to the probabilistic setting by defining a monadic semantics in the style of \obliv in which
each \mpc rule corresponds to a \lang rule that produces a point distribution (⸨‹return›⸩ in
the Giry monad). For example, here's two candidate rules for introducing pairs and performing
a zero-information revelation.

P⁅ Rː⦗Allyn-ST-Pair⦘
   R⁅{Aːrcl
      A⁅ v₁ ⧼=⧽ γ(x₁)↙⸤m⸥
      A⁃ v₂ ⧼=⧽ γ(x₂)↙⸤m⸥
      A⁆}
      -------------------------------------------
      γ ⊢⸤m⸥ δ,⟨x₁,x₂⟩ ↪ ‹return›(δ,⟨v₁,v₂⟩@m)
      R⁆
P⁃ Rː⦗Allyn-ST-Reveal-True⦘
   R⁅{Aː[b]rcl
      A⁅ p@m          ⧼=⧽ γ(x₁)↙⸤m⸥
      A⁃ q@m          ⧼=⧽ γ(x₂)↙⸤m⸥
      A⁃ i⸢⦑\faThumbsUp⦒ ; ⦑enc⦒⋕p⸣@p ⧼=⧽ γ(x₃)↙⸤p⸥
      A⁆}
   R⁃{Aː[b]rcl
      A⁅ q ⧼≠⧽ ∅
      A⁃ m ⧼=⧽ p ∪ q
      A⁆}
      -------------------------------------------
      γ ⊢⸤m⸥ δ,⦑reveal⦒⸤⦑true⦒⸥[x₁→x₂]␣x₃ ↪ ‹return›(δ,i⸢⦑\faThumbsDown⦒⸣@q)
   R⁆
P⁆

We expect most rules to follow this adaptation pattern. However, one design decision
that must be made is whether or not to constrain the mode when evaluating a uniform
expression, ⸨⦑unif⦒⸢ρ⸣␣μ⸩. The simplest design is to force the mode to be a singleton.
If we want multiple parties to agree on a uniform value, it must be explicitly sent (shared)
with those parties. Alternatively, we could allow ⸨⦑unif⦒⸢ρ⸣␣μ⸩ to be evaluated in an arbitrary
mode ⸨m⸩. However, this is more challenging to realize in a distributed implementation since the
entropy on all parties has to be identical throughout execution to ensure that the uniform values
they generate are also identical. The advantage of this approach is that it is (potentially) faster
since there is no communication needed among the parties.

As we will see, most of the complexity of \lang crops up in the design of the type system, and the
proof strategies required for~\nameref{thm:lang-pmto} and~\nameref{thm:lang-simulation}.

\subsection{Type System}
\label{subsec:proposal-design-types}

There are three questions which must be addressed to effectively
design \lang's type system. First, what should the information-flow
labels be? Second, how do we type-check code which uses party sets as values?
Third, how do we type-check code which uses probability regions as values?

The information-flow labels have to be determined before obliviousness
can be enforced. The traditional public and secret labels don't make
sense in the MPC setting. A clear notion of the information-flow
lattice and labels is necessary because it is the foundational
definition of trust. If this is wrong, then security
theorems like non-interference and obliviousness are meaningless.

What should the typing rule for parallel execution ⸨⦑par⦒[e₁]␣e₂⸩ be?
To determine how the mode should be modified, we need to know the value
of the party set ⸨p⸩ which results from evaluating ⸨e₁⸩. This is what
makes type-checking code which uses party sets as values difficult.

Although not as clearly necessary, we also anticipate that we will
need to type-check code which uses probability regions as values.
For example, expressions like ⸨⦑unif⦒⸢e⸣␣μ⸩ where ⸨e⸩ evaluates
to a probability region ⸨ρ⸩.

\paragraph*{Information Flow Labels}

The traditional information-flow lattice for confidentiality is
typically the simple lattice denoting privileged (secret) and
unprivileged (public) entities. In this lattice, public is
ordered lower than secret. This lattice works well for traditional
information flow research in which there is a conceptual untrusted
client and trusted server. However, in MPC we typically think of
the various parties who are jointly computing as each having their
own secrets which they want to keep private from each other. In
that sense, it makes sense for the information-flow lattice to
be the (dual) powerset lattice over the universe of parties. The
bottom of the lattice is the universe of all parties, analogous
to ``public.'' The top of the lattice is the empty set, analogous
to ``secret.''

If we ignore MPC for a moment, then we can see that the location
annotations on types and values serve as static and dynamic information
flow labels. When a value is introduced, the mode determines its location (label).
When a value is eliminated, the mode is checked against its location (label).
We can view the location checks as pulling ``double-duty'' as information-flow
checks.

However, we must decide how MPC and secret-share types should be integrated into
the information-flow lattice. Our first attempt at such a lattice was to ``stack''
two powerset lattices, one for cleartext values and one for encryted values. We
use the location annotation as the label for cleartext values, and we use the
encryption annotation ⸨⦑enc⦒⋕m⸩ as the label for encrypted values. We then
sum these two lattices by asserting that top in the cleartext lattice ⸨⊤⸢‹c›⸣⸩
(cleartext values with empty location) are ordered below the bottom of the
encryption lattice ⸨⊥⸢‹e›⸣⸩ (encrypted values among all parties in the universe).
This approach breaks down when considering the least upper bound of a cleartext
label ⸨p⸢‹c›⸣⸩ and an encrypted label ⸨q⸢‹e›⸣⸩ where ⸨q ⊂ p⸩. The least upper
bound of two such labels is ⸨p⸢‹e›⸣⸩ according to the candidate lattice. However,
this doesn't match our intuition. A cleartext value known to ⸨p = \{A,B,C\}⸩ cannot
be used in a context that expects a value encrypted amongst ⸨q = \{A,B\}⸩.

We are confident that the location and encryption tags can be used to derive an appropriate
information-flow label, but it is ongoing work to figure out what the labels should be. Doing
so is a pre-requisite for the design of our type system.

\paragraph*{Polymorphic Types (Party Sets and Probability Regions)}

The most expressive type system design which handles first-class party sets
and probability regions is a dependent type system. Types would be allowed to
depend fully on party sets and probability regions (but not other kinds of values).
The advantage of this approach is that it is maximally expressive, meaning that
many programs can be made to type-check. The disadvantages are that it significantly
increases the burden on the language designer and the programmer. The language designer
must deal with issues such as normalization of open (symbolic) terms. The programmer
is often responsible for proving to the type-checker that a term has the appropriate type.
For example, if a given term ⸨e₂⸩ must be in mode ⸨m'⸩ in order to type-check, then the
programmer must prove that ⸨m ⊓ p = m'⸩ where ⸨p⸩ is the value of ⸨e₂⸩ at runtime.

A more ergonomic approach is to instead use Liquid types, which are refinement types
where the refinments fall into a decidable theory. This approach is very expressive,
while eliminating the burden on the language designer and the programmer. The language
designer can reuse the metatheory of the decidable fragment (e.g. Presburger arithmetic)
and defer to a decision procedure in typing rules. Likewise, the programmer does not receive
any proof obligations from the type-checker. This same approach was taken by Wysteria~\cite{todo},
where refinements are singletons (a reflection of a party set literal to the type level),
subsets, or equality.

\ins{TODO: say what will be different from Wysteria ... I imagine the differences will crop up
  due to us using intersection rather than subset, and allowing case analysis on party sets (which
  Wysteria doesn't) ... can Wysteria ever shrink a party set value? Or are they always monotonically
  increasing?}

\ins{TODO: Give a rough candidate for PMTO\%, where does the proof strategy need adapting?} \\

\begin{theorem}[PMTO\%]\label{thm:lang-pmto}
\end{theorem}

\ins{TODO: Discuss connection between PMTO\% and gradual release.} \\

\ins{TODO: Make sure to mention SCVM as prior work that establishes something like MTO\%. Also
  mention that it is much less useful as a theorem having proved Simulation.} \\

\subsection{Distributed Semantics}
\label{subsec:proposal-design-dist}

We expect the distributed semantics of \lang to be a straight-forward combination of the
distributed semantics of \mpc and \obliv. Since none of the expression forms added to
\lang require synchronization, we anticipate that these will be handled by \lang's version
of the *⦗DS-Step⦘ rule in~\Cref{fig:mpc-dist}. A \lang distributed configuration, ⸨ ⇡~*C ⸩,
will be a finite map from parties to \emph{distributions} of local configurations ⸨⇡.ς⸩.
We need not lift the distribution over the entire finite map, since there are no expressions
which are both probabilistic and synchronizing.

Here's an candidate adaptation of the *⦗DS-Step⦘ rule, according to the description above. As
a notational convenience, we notate distributions of local configurations as ⸨⇡^ς⸩.

P⁅ Rː⦗Allyn-DS-Step⦘
   R⁅ ⇡^ς —→⸤A⸥ ⇡^{ς'}
      ----------------
      ❴A ↦ ⇡^ς ❵ ⊎ ⇡~*C ↝ ❴A ↦ ⇡^{ς′} ❵ ⊎ ⇡~*C
   R⁆
P⁆

The local step relation from \mpc, ⸨⇡^ς —→⸤A⸥ ⇡^ς⸩, is adapted in the same way as described
in~\Cref{para:mpc-design-dist-sem-nonsync}, and then the relation is lifted monadically.

We give a sketch of~\Cref{thm:lang-simulation} below, but the notation is intended to be evocative rather
than precise. We denote distributions using~⸨⇡~*{}⸩. Generally, we expect to have to lift most definitions from~\Cref{subsec:mpc-design-sim}
to a probabilistic variant. What does it mean for a distribution of configurations to be terminal? What is the definition of slicing, ⸨‗↯⸩,
over distributions of configurations? Does the natural lifting relate sequential and distribution configurations appropriately? These
are all questions we will investigate.

\begin{theorem}[Simulation (\lang)]\label{thm:lang-simulation}
    If ⸨ς ⇡~*{—→}⋆ ⇡~*{ς'}⸩ then ⸨⇡~*{ς'}⸩ is a terminal configuration and ⸨⇡~*C = ⇡~*{ς'}↯⸩ ⸨⟺⸩ ⸨ς↯ ⇡~*{↝}⋆ ⇡~*C ⸩ and ⸨ ⇡~*C ⸩ is a terminal configuration.
\end{theorem}

If we phrase the sequential semantics as a \emph{deterministic} semantics over \emph{distributions} of configurations
(rather than a \emph{probabilistic} semantics over \emph{individual} configurations) we may be able to recover much
of the proof strategy used in the proof of~\nameref{thm:mpc-simulation}. Since the distributions in the distributed
semantics are over local configurations, we anticipate that the semantics will still satisfy confluence up to equivalence
of distributions. Since the sequential semantics will still be deterministic (over distributions), we anticipate that the
proof of~\nameref{thm:lang-simulation} will also be easily adapted. The bulk of the work, then, will go to proving that
\lang satisfies Forward Simulation (appropriately adapted to distributions).

\section{Implementation}
\label{sec:proposal-impl}

Our implementation of \lang will build on the existing Haskell interpreter for \mpc, \system, described in~\Cref{sec:mpc-impl}.
The interpreter will need to be extended with support for a cryptographically secure PRNG. This will serve as a secure foundations for
generating uniform, random distributions which are suitable for use in cryptographic protocols. One possibility is to link the interpreter
against OpenSSL and use AES in CTR mode.

Perhaps the biggest implementation effort will be adding a sufficiently expressive type checker to the interpreter. As discussed
in~\Cref{subsec:proposal-design-types}, the type system will require polymorphism over probability regions and party sets. To
type check interesting programs, we will also need to support constraints over regions and party sets. To implement such a type system
will require linking against an SMT solver like Z3~\cite{} to check that constraints are satisfiable. All that being said, we have
implemented a type checker for \obliv as part of the \textsc{OblivML} interpreter so we anticipate that consulting this code
will make the implementation effort much easier.

As discussed in~\Cref{sec:mpc-impl}, the \system interpreter already has an extensive standard library, test suite, and benchmark
suite. We will evaluate how much of this existing \system code can be made to typecheck (i.e. by adding type annotations where necessary).
However, none of this existing code uses randomness. So, we anticipate that this will primarily evaluate the expressiveness of our
typing rule for party set polymorphism and associated constraints. To evaluate the expressiveness of code which uses randomness,
we'll implement a new set of benchmark programs. Our hope is to implement and type check One-Time Pad, Circuit ORAM,
Binary Search (on top of Circuit ORAM), and an optimized QuickSort using permutation~\cite{hamada}. All four of these programs
will rely on the type system to ensure that certain revelations leak zero information. The inclusion of One-Time
Pad and optimized QuickSort will provide evidence that our type system generalizes to programs beyond ORAM.

We will evaluate the efficiency of \lang using a methodology identical to the one described in \Cref{sec:mpc-impl} on the
benchmark programs described above (OTP, Circuit ORAM, Binary Search, and optimized QuickSort). We will also repeat the MPC
benchmarks to see how much performance improvement we achieve through static typing. We expect at least a modest improvement,
since we can omit all runtime checks for well-typed programs.
