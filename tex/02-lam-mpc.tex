\chapter{\mpc, A Language for Concise MPC}
\label{ch:lam-mpc}

In~\cref{sec:background-symphony} we walked through the sequential interpretation of the Millionaire's Problem in \mpc. Now that we have
the intution, let's look at the formal description of \mpc. Having a formal model will allow us to state the Simulation property of the
language precisely.

\ins{todo: refer to simulaion as theorem}

\section{Overview}
\label{sec:lam-mpc-overview}

The syntax of \mpc is presented in \Cref{fig:mpc-syntax}. \mpc types comprise a
series of standard types---integers, booleans functions, and pair types---augmented
with two additional elements.

F⁅
\begingroup
\setlength\arraycolsep{0pt} % default is 6pt
\smaller
D⁅
M⁅
Aːrcrcl@{␠}l
A⁅ b     ⧼∈⧽ 𝔹            ⧼ ⧽                                & ⟪booleans⟫
A⁃ i     ⧼∈⧽ ℤ            ⧼ ⧽                                & ⟪integers⟫
A⁃ A,B,C ⧼∈⧽ ‹party›      ⧼ ⧽                                & ⟪parties⟫
A⁃ m,p,q ⧼∈⧽ ‹party-set›  ⧼≜⧽ ℘(‹party›) ⩴ ❴A,…,A❵          & ⟪sets of parties⟫
A⁃ ψ     ⧼∈⧽ ‹prot›       ⧼⩴⧽ ⋅                             & ⟪cleartext⟫
A⁃       ⧼ ⧽              ⧼¦⧽ ⦑enc⦒⋕m                        & ⟪encrypted⟫
A⁃ μ     ⧼∈⧽ ‹base-type›  ⧼⩴⧽ ⦑int⦒                         & ⟪integer type⟫
A⁃       ⧼ ⧽              ⧼¦⧽ ⦑bool⦒                         & ⟪boolean type⟫
A⁃ σ     ⧼∈⧽ ‹loc-type›   ⧼⩴⧽ μ⸢ψ⸣                          & ⟪protocol type⟫
A⁃       ⧼ ⧽              ⧼¦⧽ τ →ₘ τ                         & ⟪function type⟫
A⁃       ⧼ ⧽              ⧼¦⧽ τ × τ                          & ⟪pair type⟫
A⁃ τ     ⧼∈⧽ ‹type›       ⧼⩴⧽ σ@m                           & ⟪located type⟫
A⁃ x,y,z ⧼∈⧽ ‹var›        ⧼ ⧽                                & ⟪variables⟫
A⁃ ⊙     ⧼∈⧽ ‹binop›      ⧼ ⧽                                & ⟪binary operations (e.g., plus, times)⟫
A⁃ e     ⧼∈⧽ ‹expr›       ⧼⩴⧽ x                             & ⟪variable reference⟫
A⁃       ⧼ ⧽              ⧼¦⧽ i                              & ⟪integer literal⟫
A⁃       ⧼ ⧽              ⧼¦⧽ b                              & ⟪boolean literal⟫
A⁃       ⧼ ⧽              ⧼¦⧽ e ⊙ e                          & ⟪binary operation⟫
A⁃       ⧼ ⧽              ⧼¦⧽ ⦑if⦒␣e␣⦑then⦒␣e␣⦑else⦒␣e       & ⟪atomic conditional⟫
A⁃       ⧼ ⧽              ⧼¦⧽ e ¿ e ◇ e                      & ⟪atomic multiplexor⟫
A⁃       ⧼ ⧽              ⧼¦⧽ ⟨e,e⟩                          & ⟪pair creation⟫
A⁃       ⧼ ⧽              ⧼¦⧽ πᵢ␣e                           & ⟪pair projection (⸨i ∈ ❴1,2❵⸩)⟫
A⁃       ⧼ ⧽              ⧼¦⧽ λ⸤z⸥x.␣e                       & ⟪(recursive) function creation⟫
A⁃       ⧼ ⧽              ⧼¦⧽ ⦑read⦒␣μ                       & ⟪read int input⟫
A⁃       ⧼ ⧽              ⧼¦⧽ ⦑write⦒␣e                      & ⟪write output⟫
A⁃       ⧼ ⧽              ⧼¦⧽ ⦑share⦒[p→p]␣e                 & ⟪share encrypted value⟫
A⁃       ⧼ ⧽              ⧼¦⧽ ⦑reveal⦒[p]␣e                  & ⟪reveal encrypted value⟫
A⁃       ⧼ ⧽              ⧼¦⧽ e␣e                            & ⟪function elimination⟫
A⁃       ⧼ ⧽              ⧼¦⧽ ⦑par⦒[p]␣e                     & ⟪parallel execution⟫
A⁃       ⧼ ⧽              ⧼¦⧽ ⦑let⦒␣x=e␣⦑in⦒␣e               & ⟪let binding⟫
A⁆
M⁆
D⁆
\endgroup
\caption{\mpc Syntax}
\label{fig:mpc-syntax}
F⁆

The first exotic feature is \emph{located types}. These types are located
in the sense that they can be manipulated only at particular principals.
The metavariable ⸨m⸩ represents a set of parties, and a type is written ⸨σ@m⸩. The \mpc expression
⸨⦑par⦒[p]␣e⸩ determines the locatedness. It says that ⸨e⸩ may be computed at
parties ⸨p⸩ in parallel (hence the syntax ⸨⦑par⦒⸩). That is, every
party ⸨A ∈ p⸩ may evaluate ⸨e⸩. We say ``may'' here because
nesting such an expression in another ⸨⦑par⦒⸩ could shrink the set of
parties. For example, ⸨e⸩ in ⸨⦑par⦒[p]␣(⦑par⦒[q]␣e)⸩ will be
evaluated by ⸨p ∩ q⸩; if this intersection is ⸨ø⸩ then ⸨e⸩
is essentially dead code.

We call the set of parties ⸨m⸩ computing an expression in parallel
the \emph{mode}. We say the parties ⸨A ∈ m⸩ are \emph{present}
for a computation. The semantics of many constructs depends on the
mode. A number $i$ created in mode $m$ (having type ⸨⦑int⦒@m⸩) is
known only to parties $A \in m$. This means that adding two numbers
located at $p$ can only be done in a mode $m$ such that
$m \subseteq p$; if the mode contained additional parties
$A \not\in p$ then they wouldn't know what to do; such states will be
stuck in our semantics. The same goes for pairs and functions.
The ⸨⦑read⦒␣μ⸩ and ⸨⦑write⦒␣e⸩ expressions perform local
I/O and so can only be run in a mode with a single party.

On the other hand, it is possible that a variable $x$ is in scope for
$A$, but maps to a value only known to $B$. Party $A$ can
still manipulate a placeholder for $x$ (e.g., to store it in a
datastructure or pass it to a function) but may never compute with its
contents (e.g., add to it or branch on it). This approach
simplifies the design of the language at the cost of missing some
(unlikely, but ultimately harmless) logic errors.

Generally speaking, \mpc's design aims to ensure that any expression
$e$ of type $\sigma @ m$ will have the \emph{same run-time value} at
each $A \in m$. This is a key invariant underpinning \mpc's sequential
interpretation.

The second exotic feature is \emph{secret share types}.
Base types are annotated with a
\emph{protocol} $\psi$ that indicates whether they are either
cleartext ($\cdot$) or are \emph{encrypted} and shared among parties
$p$. For concreteness, we imagine the encryption protocol is
Goldreich, Widgerson, and Micali (GMW)~\citeyear{STOC:GolMicWig87},
and values of this type are \emph{secret shares}, but the formal
semantics is agnostic to the cryptographic details. Note that we
often just write ⸨⦑int⦒⸩ for cleartext integer types, to reduce
clutter (i.e., eliding the $\cdot$).

The \mpc expression ⸨⦑share⦒[p→q]␣e⸩ directs $p$ (required to be a singleton party set) to
create secret shares of $e$, an integer, and distribute the shares to
each party in
$q$. All parties $p \cup q$ must be present. The resulting value
has type ⸨⦑int⦒⸢⦑enc⦒⋕q⸣@q⸩. This type reads “an integer, encrypted
(i.e., secret shared) between parties ⸨q⸩, and accessible to parties
⸨q⸩”. The duplication of ⸨q⸩ may seem redundant, but they may differ
in other contexts. The first ⸨q⸩ represents \emph{who has the
shares} (determined when the share is created), and the second ⸨q⸩
represents \emph{who has access to this value} (determined by the
enclosing ⦑par⦒ blocks). If this encrypted value flows to a part of
the program only executed by ⸨q′ ⊂ q⸩, then it will not be possible to
recombine shares, since not all parties ⸨q⸩ will be present.

Parties $q$ can all mutually compute on a shared, encrypted value
using ⸨e₁ ⊙ e₂⸩ and ⸨e ¿ e₁ ◇ e₂⸩. The latter is a
multiplexor: we select between ⸨v₁⸩ and ⸨v₂⸩, the results of evaluating ⸨e₁⸩ and ⸨e₂⸩,
based on whether $e$ is ⸨true⸩ or ⸨false⸩. The former models binary operations over numeric and boolean types.
When operating on encrypted values, both the multiplexor and some binary
operation expressions will necessitate \emph{communication} in the
distributed semantics, and an actual implementation will use an
underlying MPC protocol to implement the computation over the encrypted value.
All parties to which a share was sent must be present when computing on it. E.g., for numbers of type
⸨⦑int⦒⸢⦑enc⦒⋕{A,B}⸣⸩ to be added, both $A$ and $B$ must be
present. Indeed, it must be \emph{exactly} these parties which are
present; we do not allow more, since other parties would not be
able to carry out the operation (they don't have access to the
share).

An MPC is completed by invoking ⸨⦑reveal⦒[q]␣e⸩. This takes a share
(among some set of parties $p$) and converts it to cleartext,
sharing the result among parties $q$. Doing so requires that all of
$p \cup q$ are present so that the shareholders
can agree to send the value, and the result-receivers are ready to
receive it.

The notion of locatedness is crucial in proving that \mpc{} satisfies~\nameref{thm:mpc-simulation}.
The~\nameref{thm:mpc-simulation} property is what allows us to view the language as \textbf{abstractly sequential}. The
next section is dedicated to explaining this property, and highlighting some of its consequences.

\section{Simulation}
\label{sec:lam-mpc-simulation}

\subsection{Sequential Semantics}

\ins{TOOD: put actual figure here?}

The sequential semantics is defined by a judgement ⸨ς → ς⸩ which says that
one configuration, ⸨ς ⩴ m,γ,κ,e⸩, can step to another. A configuration
⸨ς⸩ is a 4-tuple comprising the current mode ⸨m⸩, environment ⸨γ⸩, stack
⸨κ⸩, and expression ⸨e⸩. Environments ⸨γ ⩴ var ⇀ value⸩, map variables to
values. A stack ⸨κ⸩ is a list of frames ⸨⟨⦑let⦒␣x=□␣⦑in⦒␣e¦m,γ⟩⸩, where
⸨⊤⸩ represents the empty stack. Values, ⸨v ⩴ u@m␣|␣★⸩, are either located
values ⸨u⸩ at mode ⸨m⸩ or else are opaque, ⸨★⸩. A located value, ⸨u⸩,
contains all standard values mutually defined on ⸨v⸩, e.g. a pair value is ⸨⟨v, v⟩⸩.
Base values are tagged with a protocol ⸨ψ⸩. For example, a
pair of shares among ⸨\{A,B\}⸩ is ⸨⟨i₁⸢⦑enc⦒⋕\{A,B\}⸣@\{A,B\},i₂⸢⦑enc⦒⋕\{A,B\}⸣@\{A,B\}⟩@\{A,B\}⸩.

Finally, the semantics heavily relies on a metafunction, ⸨‗↙ₘ ∈ ‹value› → ‹value›⸩, which \emph{(re)locates} a
value to a scope with parties ⸨m⸩ present. For located values ⸨u@p⸩, ⸨‗↙ₘ⸩
relocates them to ⸨p ∩ m⸩, unless the intersection is empty in which case the
value is inaccessible, so it becomes ⸨★⸩. Relocating is a deep operation;
⸨u@p↙ₘ⸩ also relocates the contents ⸨u⸩ to ⸨u↙ₘ⸩, which recurses over
the sub-terms of ⸨u⸩. This step has no effect on closures, though: the closure's
environment's variables get relocated when they are referenced.

The semantics ensure the whenever a located value ⸨u⸩ is introduced, it is tagged with
the appropriate mode ⸨m⸩ from the current configuration ⸨ς⸩. The relocation metafunction
is used on variable access, to relocate a value to the mode of the current lexical scope.
For example, the program ⸨⦑let⦒␣x = ⦑par⦒␣[A,B] 10␣⦑in⦒␣⦑par⦒␣[A]␣x⸩ should evaluate to
⸨10@{A}⸩. Likewise, when a value ⸨u@p⸩ is eliminated, we check that all necessary parties
are present ⸨m ⊆ p⸩. These two patterns ensure that the sequential semantics will get stuck
whenever a distributed deployment would get stuck. For example, in the program
⸨⦑par⦒␣[A,B]␣(⦑par⦒␣[A]␣10) + (⦑par⦒␣[B]␣20)⸩, the introduction rules for integers ensure that
the inner ⸨⦑par⦒⸩ expressions will reduce to ⸨10@\{A\}⸩ and ⸨20@\{B\}⸩. The elimination rule, ⸨+⸩,
will check ⸨\{A,B\} ⊆ \{A\}⸩ and ⸨\{A,B\} ⊆ \{B\}⸩ which will fail, causing the program to get stuck.

\subsection{Distributed Semantics}

The distributed semantics is defined by a judgement ⸨D ↝ D⸩ which says that
one distributed configuration, ⸨D ∈ ‹party› ⇀ ‹local-config›⸩, can step to another.
A configuration, ⸨D⸩, is a partial mapping between parties and their local
view of execution. A ⸨‹local-config›⸩ is just like a sequential configuration, ⸨ς⸩,
except that the components contain local values, ⸨⇡.v⸩. Local values are just like
sequential values, except that location annotations ⸨‗@m⸩ have been stripped away.
For example, a pair of shares among ⸨\{A,B\}⸩ is ⸨⟨i₁⸢⦑enc⦒⋕\{A,B\}⸣,i₂⸢⦑enc⦒⋕\{A,B\}⸣⟩⸩.

For most expressions, the distributed semantics simply steps an individual party independently of
the rest. To step an individual party, ⸨P⸩, we execute a step on its local configuration, ⸨D(P)⸩.
These individual steps look very similar to sequential steps, except that mode annotation is not added
or checked. A local configuration still has a mode component, because we do need to check that the mode
agrees with, for example, the protocol tag ⸨⦑enc⦒⋕q⸩. When locally stepping an expression ⸨⦑par⦒␣[P]␣…⸩
we additionally check that ⸨P ∈ p⸩ where ⸨P⸩ is the whose local configuration we are stepping.

The only expressions which do not step on an
invididual party are the \emph{synchronizing} expressions, ⸨⦑share⦒⸩ and ⸨⦑reveal⦒⸩. These expressions
step by transferring data from one party to the other(s), and so the rules maniuplate multiple local
configurations at once. In both the sequential and distributed rules we require that (1) the sharer,
⸨p⸩, be a singleton ⸨|p| = 1⸩, (2) the sharees, ⸨q⸩, be non-empty ⸨q ≠ ø⸩, and (3) that only the
sharer and sharees are present ⸨m = p ∪ q⸩. In distributed mode, we move the shared value from ⸨p⸩
to ⸨q⸩ by pulling ⸨⇡.v⸩ from ⸨D(p)⸩ and updating all the local configurations ⸨D(q)⸩.

\subsection{Correspondence}

The key observation that makes these semantics agree is that if the sequential semantics
gets stuck due to an elimination form with a failing mode check then the distributed semantics
will get stuck due to an elimination form attempting to inspect the opaque value ⸨★⸩. For example,
let's return to the example ⸨⦑par⦒␣[A,B]␣(⦑par⦒␣[A]␣10) + (⦑par⦒␣[B]␣20)⸩. We saw that this expression
gets stuck in the sequential semantics due to a failure of both mode checks when evaluating ⸨+⸩:
⸨\{A,B\} ⊈ \{A\}⸩ and ⸨\{A,B\} ⊈ \{B\}⸩. In the distributed semantics, we will execute each party
independently, eventually arriving at ⸨10 + ★⸩ on $A$ and ⸨★ + 20⸩ on $B$. Both parties will get
stuck, and the entire distributed configuration will be stuck.

We can state our formal correspondence between the sequential and distributed semantics formally
as~\nameref{thm:mpc-simulation}. This theorem says that for any well-typed sequential configuration,
⸨ς⸩, ⸨ς⸩ successfully terminates as ⸨ς'⸩ if and only if the corresponding distributed configuration
⸨ς↯⸩ successfully terminates as ⸨ς'↯⸩. The ⸨‗↯⸩ metafunction takes a sequential configuration to an
appropriate distributed configuration. We call it this metafunction \emph{slicing} since it ``slices''
away all of the mode annotations appropriately, leaving a distributed configuration in which each
party is mapped to a local configuration containing only local values which that party knows.

The forward direction of~\nameref{thm:mpc-simulation} is proved by induction. The backward direction
requires that we first establish that the distributed semantics enjoys \emph{confluence}. Confluence
says that any executions beginning with the same configuration must eventually arrive at a shared
configuration. For the distributed semantics, this means that for any distributed configuration ⸨D⸩,
if ⸨D ↝⋆ D₁⸩ and ⸨D ↝⋆ D₂⸩ then there must be a ⸨D'⸩ such that ⸨D₁ ↝⋆ D'⸩ and ⸨D₂ ↝⋆ D'⸩. The backward
direction can then be established by appealing to the forward direction, using confluence to rule out the
possibility of rogue executions.

\begin{theorem}[Forward Simulation] \label{thm:mpc-simulation}
  ⸨∀␣⊢ ς⸩, ⸨ς —→⋆ ς'⸩ and ⸨ς'⸩ is terminal ⸨⟺⸩ ⸨ς↯ ↝⋆ ς'↯⸩ and ⸨ς'↯⸩ is terminal.
\end{theorem}

Intuitively,~\nameref{thm:mpc-simulation} tells us that all well-typed, terminating programs
will give us the same answer whether they are executed in the sequential or distributed semantics.
This makes precise what it means for a language to be a \textbf{abstractly sequential}. Programmers
can reason about the behavior of their program without considering all the possible interleaved
executions that a deployment may encounter.

\section{Symphony: An Interpreter for \mpc}

\ins{TODO}
