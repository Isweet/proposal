\chapter{\obliv, A Language for Probabilistically Oblivious Computation}
\label{ch:lam-obliv}

An oblivious computation is one that is free of direct and indirect information leaks, e.g., due to observable differences
in timing and memory access patterns. In this chapter, we'll give a formal description of \obliv, a core language whose
type system enforces obliviousness. We show that \obliv satisfies~\nameref{thm:obliv-pmto}, a probabilistic variant of obliviousness.
We also show that \obliv's type system is expressive enough to typecheck an asymptotically optimal client-server ORAM implementation.

\section{Overview}
\label{sec:lam-obliv-overview}

The syntax of \obliv is presented in \Cref{fig:obliv-syntax}. \obliv types comprise a
series of standard types---integers, booleans, functions, and pair types---as well as
a special type for secret, uniform distributions of booleans.

F⁅
\begingroup
\setlength\arraycolsep{0pt} % default is 6pt
\smaller
D⁅
M⁅
Aːrcrcl@{␠}l
A⁅ b     ⧼∈⧽ 𝔹           ⧼ ⧽                               & ⟪booleans⟫
A⁃ i     ⧼∈⧽ ℤ           ⧼ ⧽                               & ⟪integers⟫
A⁃ ℓ     ⧼∈⧽ ‹label›     ⧼⩴⧽ ‹P› ¦ ‹S›                    & ⟪public and secret⟫
A⁃       ⧼ ⧽ 𝑚𝑐3c{⟪(«where» ⸨‹P›⊏‹S›⸩)⟫}                   & ⟪security labels⟫
A⁃ ρ     ⧼∈⧽ R           ⧼ ⧽                               & ⟪probability region⟫
A⁃ μ     ⧼∈⧽ ‹base-type› ⧼⩴⧽ ⦑int⦒                        & ⟪integer type⟫
A⁃       ⧼ ⧽             ⧼¦⧽ ⦑bool⦒                        & ⟪boolean type⟫
A⁃ τ     ⧼∈⧽ ‹type›      ⧼⩴⧽ μ⸤ℓ⸥⸢ρ⸣                       & ⟪non-random base type⟫
A⁃       ⧼ ⧽             ⧼¦⧽ ⦑½⦒μ⸤‹S›⸥⸢ρ⸣                  & ⟪secret uniform bool⟫
A⁃       ⧼ ⧽             ⧼¦⧽ τ × τ                          & ⟪tuple⟫
A⁃       ⧼ ⧽             ⧼¦⧽ τ → τ                          & ⟪function⟫
A⁃ x,y,z ⧼∈⧽ ‹var›       ⧼ ⧽                                & ⟪variables⟫
A⁃ ⊙     ⧼∈⧽ ‹binop›     ⧼ ⧽                                & ⟪binary operations (e.g., plus, times)⟫
A⁃ e     ⧼∈⧽ ‹expr›      ⧼⩴⧽ x                             & ⟪variable reference⟫
A⁃       ⧼ ⧽             ⧼¦⧽ i⸤ℓ⸥                           & ⟪integer literal⟫
A⁃       ⧼ ⧽             ⧼¦⧽ b⸤ℓ⸥                           & ⟪boolean literal⟫
A⁃       ⧼ ⧽             ⧼¦⧽ e ⊙ e                          & ⟪binary operation⟫
A⁃       ⧼ ⧽             ⧼¦⧽ ⦑if⦒␣e␣⦑then⦒␣e␣⦑else⦒␣e       & ⟪conditional⟫
A⁃       ⧼ ⧽             ⧼¦⧽ e ¿ e ◇ e                      & ⟪multiplexor⟫
A⁃       ⧼ ⧽             ⧼¦⧽ ⟨e,e⟩                          & ⟪pair creation⟫
A⁃       ⧼ ⧽             ⧼¦⧽ πᵢ␣e                           & ⟪pair projection (⸨i ∈ ❴1,2❵⸩)⟫
A⁃       ⧼ ⧽             ⧼¦⧽ λ⸤z⸥x⍪ e                       & ⟪(recursive) function creation⟫
A⁃       ⧼ ⧽             ⧼¦⧽ e␣e                            & ⟪function elimination⟫
A⁃       ⧼ ⧽             ⧼¦⧽ ⦑let⦒␣x = e␣⦑in⦒␣e             & ⟪variable binding⟫
A⁃       ⧼ ⧽             ⧼¦⧽ ⦑unif⦒⸢ρ⸣␣μ                    & ⟪uniform, random sample in region⟫
A⁃       ⧼ ⧽             ⧼¦⧽ ⦑cast⦒⸤l⸥␣e                    & ⟪cast from uniform⟫
A⁆
M⁆
D⁆
\endgroup
\caption{\obliv Syntax}
\label{fig:obliv-syntax}
F⁆

The only exotic feature of \obliv is the inclusion of \emph{uniform types}. A
value of uniform type ⸨⦑½⦒ μ⸤‹S›⸥⸢ρ⸣⸩ is always labeled secret (⸨‹S›⸩), and represents
a uniform distribution over the base type ⸨μ⸩. The \emph{probability region}, ⸨ρ⸩, is
a static name for a collection of uniform values, and the base values that may be derived
from them. Uniform values are associated with a region ⸨ρ⸩ when they are created with
⸨⦑unif⦒⸢ρ⸣␣μ⸩. The probability region annotation is propagated from the uniform type
⸨⦑½⦒μ⸤‹S›⸥⸢ρ⸣⸩ to base type ⸨μ⸤‹S›⸥⸢ρ⸣⸩ when typing a secret cast, ⸨⦑cast⦒⸤‹S›⸥␣e⸩.
Both literals ⸨b⸤ℓ⸥⸩ and public casts, ⸨⦑cast⦒⸤‹P›⸥␣e⸩, are assigned the probability
region ⸨⊥⸩. We say that a value is ``in region ⸨ρ⸩'' if it has type ⸨⦑½⦒μ⸤‹S›⸥⸢ρ⸣⸩
or ⸨μ⸤ℓ⸥⸢ρ⸣⸩.

Probability regions form a join semi-lattice. The type system maintains the invariant that
values of in region ⸨ρ⸩ are probabilistically independent of values in region ⸨ρ'⸩, provided
that ⸨ρ' ⊏ ρ⸩. For example, the typing rule for
multiplexors says that if ⸨e₁ : μ⸤ℓ⸥⸢ρ₁⸣⸩, ⸨e₂ : ⦑½⦒μ⸤‹S›⸥⸢ρ₂⸣⸩, and ⸨e₃ : ⦑½⦒μ⸤‹S›⸥⸢ρ₃⸣⸩,
then we must check ⸨ρ₁ ⊏ ρ₂⸩ and ⸨ρ₁ ⊏ ρ₃⸩ to ensure that the result of the multiplexor
is also uniform. Intuitively, we are saying that if the guard of the multiplexor is probabilistically
independent of each of the branches, and the branches are uniform, then the result
of the multiplexor is also uniform.

\section{Obliviousness}
\label{sec:lam-obliv-obliviousness}

The semantics of \obliv are defined by a judgement ⸨ς —→ ⇡~{ς}⸩ which says that
configuration ⸨ς ⩴ γ,κ,e⸩, takes a step to the distribution of configurations ⸨⇡~{ς}⸩.
All the expressions which are deterministic (i.e. all expressions except ⸨⦑unif⦒⸢ρ⸣␣μ⸩)
produce a point distribution over the appropriate resulting configuration. For example,
the expression ⸨10⸤‹P›⸥ + 20⸤‹P›⸥⸩ evaluates to the point distribution of ⸨30⸤‹P›⸥⸩.
The uniform expression, ⸨⦑unif⦒⸢ρ⸣␣μ⸩, simply lifts the appropriate uniform distribution
over values of base type ⸨μ⸩ to configurations.

The evaluation relation ⸨e —→⋆ ⇡~{t}⸩ is defined by first inserting ⸨e⸩ into the initial
configuration, ⸨ø,⊥,e⸩, and then lifting the single-step judgement ⸨—→⸩ to distributions
of traces. This is done by standard monadic composition in the Giry monad. A trace ⸨t⸩
is a list of configurations encountered during execution.

Lastly, we must define a notion of label-equivalence over expressions and distributions
of traces. The definition of label-equivalence over expressions, ⸨e₁ ≈⸤ℓ⸥ e₂⸩, is standard.
We can define label-equivalence of trace distributions ⸨⇡~{t₁} ⇡~≈⸤l⸥ ⇡~{t₂}⸩, as follows.
We say that two trace distributions are label-equivalent if the distributions are probabilistically
after projecting according to label-equivalence on traces. The definition of label-equivalence on traces
is standard: their contents must be point-wise label-equivalent. Finally, two configurations are label-equivalent
if all their componenets are, where the low-equivalence of the environment and stack are defined recursively.

The key property of the type system which ensures that programs are oblivious is that only (conditionally) uniform
values are ever casted to public. Since uniform distributions carry no information, they are guaranteed not to leak
any information about secrets in the program. The challenge is establishing that a value is uniform, even when it may
have been duplicated (using ⸨⦑cast⦒⸤‹S›⸥⸩) and correlated with other distributions (using a multiplexor).

We formally define an oblivious language as one that satisfies~\nameref{thm:obliv-pmto}. This theorem says that two label-equivalent,
well-typed expressions, ⸨e₁⸩ and ⸨e₂⸩ evaluate to distributions of traces ⸨⇡~{t₁}⸩ and ⸨⇡~{t₂}⸩ which are also label-equivalent.
The proof is complex, involving a simulation argument with an alternative ``mixed'' semantics, and then a separate proof that
individual steps in the mixed semantics are also probabilistically oblivious.

\begin{theorem}[PMTO] \label{thm:obliv-pmto}
  If ⸨e₁ : τ⸩, ⸨e₂ : τ⸩, ⸨e₁ ≈⸤l⸥ e₂⸩, ⸨e₁ —→⋆ ⇡~{t₁}⸩, and ⸨e₂ —→⋆ ⇡~{t₂}⸩, then ⸨⇡~{t₁} ⇡~≈⸤l⸥ ⇡~{t₂}⸩.
\end{theorem}
