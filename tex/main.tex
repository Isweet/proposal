\documentclass{report}

\usepackage{epigraph}
\usepackage{colortbl}
\usepackage{xspace}
\usepackage{listings}
\usepackage{relsize}

\input{darais-latex-imports}
\input{darais-latex-macros}

\newcommand{\lang}{Allyn\xspace}
\newcommand{\mpc}{\ensuremath{\lambda_{\mathrm{MPC}}}\xspace}
\newcommand{\obliv}{\ensuremath{\lambda_{\mathrm{Obliv}}}\xspace}

\newcommand{\ins}[1]{\textcolor{red}{Ian: #1}}

%%% Macros for MPC
\newcommand{\alice}{B⸨‹Alice›B⸩\xspace}
\newcommand{\bob}{C⸨‹Bob›C⸩\xspace}

\newcommand{\alices}[1]{B⸨#1⸤A⸥B⸩}
\newcommand{\bobs}[1]{C⸨#1⸤B⸥C⸩}

\newcommand{\aliceSec}{\alices{s}\xspace}
\newcommand{\bobSec}{\bobs{s}\xspace}

\newcommand{\aliceSh}[1]{\alices{⌊#1⌋}}
\newcommand{\bobSh}[1]{\bobs{⌊#1⌋}}

\newcommand{\opaque}{A⸨★A⸩\xspace}
%%%

\definecolor{implColor}{HTML}{EA9999}
\definecolor{theoryColor}{HTML}{A4C2F4}

\begin{document}

\title{A Programming Language for Obliviousness and Secure Computation}

\author{Ian Sweet \\
  \emph{University of Maryland, College Park} \\
  \emph{ins@cs.umd.edu}}

\date{}

\maketitle

\begin{abstract}
  \emph{Secure Multiparty Computation} (MPC) and \emph{Oblivious RAM} (ORAM) have emerged as promising approaches to
  high-confidentiality computation. Traditionally, MPC programs do not allow secrets to be used as indexes to dynamically
  allocated memory. This precludes, for example, a secure binary search in which the element being accessed is considered secret.
  This restriction is relaxed by using ORAM inside of the secure computation. Unfortunately, existing languages for MPC typically
  support ORAM by ``baking it in'' to the language as a trusted primitive. We propose the design and implementation of a language
  for MPC with \emph{application-level support} for ORAM with an associated \emph{proof of security}.

  \ins{TODO: generalize from ORAM to obliviousness in general}
\end{abstract}

\tableofcontents

\chapter{Introduction}
\label{ch:intro}

\epigraph{It seems like everywhere I go \\
          The more I see, the less I know}{
            Michael Franti \& Spearhead \\
            \emph{Say Hey I Love You}}

If our adversaries were more like Michael Franti then we wouldn't need to concern ourselves with confidentiality.
Unfortunately, that isn't the world we live in. This proposal is fundamentally concerned with the right of people
to have their information kept private. Our programming languages have a duty to produce programs which honor that
right. In other words, people have a right to privacy and programming languages have a duty of confidentiality. This
proposal makes a modest step towards a future where our languages uphold their end of the bargain.

\section{Secure Multiparty Computation and Obliviousness}
\label{sec:intro-mpc}

\emph{Secure Multiparty Computation} (MPC) is a subfield of cryptography
that allows mutually untrusting parties to compute arbitrary functions over their private inputs whiel revealing nothing
except the function output. That is, MPC allows parties to work together to run programs \emph{under encryption}.

Approaches to MPC have improved significantly over the years. The first full implementation, FairPlay~\cite{}, could evaluate
only a few hundred Boolean gates pers second. Modern implementations on custom setups evaluate \emph{billions} of Boolean gates per
second~\cite{}, and cryptographers continue to reduce the cost of MPC. Despite increasing efficiency and compelling applications
(secure auctions, secure databases, collaborative machine learning, and any other imaginable application with security concerns),
MPC has not been widely adopted. One significant barrier to adoption is a lack of appropriate infrastructure. Today, it is difficult
for non-experts to understand and work in the complex distributed model that MPC requires.

In addition to lack of infrastructure, languages for MPC also fail to provide adequate support for RAM-model programming. The standard
means by which RAM-model computation is supported in MPC languages is using \emph{Oblivious RAM} (ORAM). To date, most MPC languages
simply do not provide any access to ORAM. In their 2019 SoK,~\citet{} observe that only a ``few frameworks have ORAM support, either
natively (ObliVM and SCALE-MAMBA) or via a library (Obliv-C).'' The three languages mentioned, ObliVM~\cite{}, SCALE-MAMBA~\cite{}, and
Obliv-C~\cite{}, are \textbf{probabilistic} MPC languages. According to~\citet{}, a language must have support for sampling from a uniform,
random distribution to admit asymptotically optimal ORAM. Indeed, each of these languages provides a highly optimized implementation of
Circuit ORAM~\cite{}, which is known to be asymptotically optimal (i.e. $O(\log{n})$). However, these languages could be improved by
being \textbf{abstractly centralized} and \textbf{high assurance}. An \textbf{abstractly centralized} language hides, as much as possible,
the distributed deployment of the MPC program from the programmer. Examples of existing languages which are abstractly centralized are
SCVM~\cite{}, Wysteria~\cite{}, and the recent Symphony language~\cite{}. A \textbf{high assurance} language guarantees that programs written
in the language are oblivious. We describe these properties in more detail, with examples, in Chapter~\ref{ch:background}.

In summary, we would like the following properties from our MPC language.

\begin{enumerate}
\item \label{itm:probabilistic} \underline{Probabilistic} -- \textbf{A probabilistic language, which allows sampling from uniform, random distributions, is necessary
  for implementing asymptotically optimal ORAM, oblivious algorithms, and data structures.}
  When more efficient oblivious protocols are invented (which happens every year), cryptographers would like to be able to
  implement these protocols in a MPC library.
\item \label{itm:centralized} \underline{Abstractly Centralized} -- \textbf{An abstractly centralized language, which hides the distributed deployment from the
  programmer, is necessary for lowering the barrier of entry for developers.}
  Being able to program as though the program is being executed on a single machine, without worrying about the distributed deployment
  of the program, significantly reduces the complexity of MPC programming.
\item \label{itm:assurance} \underline{High Assurance} -- \textbf{A high assurance language, which guarantees that all programs are oblivious, is necessary
  for certifying that programs do not leak information through side channels.} Without certification of obliviousness, a domain expert
  would need to manually audit the declassifications in the program which are expected to reveal no information. Such declassifications
  are common in, for example, tree-based ORAM constructions.
\end{enumerate}

\paragraph{Problem.} No existing MPC language is \textbf{(all together) probabilistic, abstractly centralized, and high assurance.}

\section{Proposed Work}

\paragraph{Hypothesis.} It is possible to design and implement a language for MPC which is \textbf{probabilistic, abstractly centralized,
  and high assurance (as defined above).}

\paragraph{Contribution: \obliv is PMTO.} In prior work, we show that ObliVM's~\cite{} type system is unsafe, and show how to fix it in
such a way that is satisfies \emph{Probabilistic Memory Trace Obliviousness} (PMTO). We prove this property for \obliv, a non-MPC language
for oblivious programming.

\paragraph{Contribution: \mpc is MTO\%.} In ongoing work, we show that Symphony's type system guarantees that MPC programs are
\emph{Memory Trace Oblivious Modulo Declassifications} (MTO\%). To our knowledge, this is the first time such a theorem has been
proved for an MPC language. The SCVM language has a proof of a related property based on crypto-style ``simulation.''

\paragraph{Contribution: Identifying PMTO\%.} We identify and define the \emph{Probabilistic Memory Trace Oblivious Modulo Declassification}
(PMTO\%) as an appropriate definition of security for probabilistic MPC languages. We also investigate and explicate the connection between
this property and other common security properties involving declassifications (such as Gradual Release~\cite{}).

\paragraph{Contribution: \lang, a probabilistic, abstractly centralized, and high assurance MPC language.} The contributions described
above have already been accomplished, and they are necessary but not sufficient for confirmation of our
hypothesis. To confirm our hypothesis, we propose the design and implementation of a new language for MPC, called \lang, which extends
the Symphony language with primitives for drawing uniform, random samples and declassifying them. By extending Symphony with sampling
from uniform, random distribuitions we will immediately satisfy property~\ref{itm:probabilistic} above. We will provide further evidence
for the utility of this property by implementing various ORAM protocols (Trivial ORAM, Circuit ORAM, and Recursive (Circuit) ORAM). We will
show that these implementations have the expected asymptotic behavior. Next, we will prove that the extensions to Symphony preserve the
existing centralized semantics by proving that the Simulation theorem continues to hold. This will establish that \lang satisfies
property~\ref{itm:centralized}. Finally, we will need to adapt the type system of Symphony to enforce PMTO\% and confirm that the case studies
(ORAM protocols) type check in the adapted type system. This will establish that \lang satisfies property~\ref{itm:assurance}. Having
accomplished this, we will have confirmed our hypothesis by constructing \lang which is probabilistic, abstractly centralized, and
high assurance.

We summarize the tasks involved in the last contribution above, and give a rough timeline for their completion. Tasks in
\colorbox{implColor}{red} require adding additional functionality to the existing Haskell implementation of \mpc.
Tasks in \colorbox{theoryColor}{blue} require adapting the formal metatheory of \mpc. \\

\ins{TODO: when proposal document is done, revisit this and update timeline.}

\begin{tabular}{|p{.30\textwidth}|p{.30\textwidth}|p{.30\textwidth}|}
  \hline
  \textbf{Task} & \textbf{Description} & \textbf{Estimated Time of Completion} \\
  \hline
  \rowcolor{implColor}
  Case Studies:
  \begin{itemize}
  \item Trivial ORAM
  \item Tree ORAM(s)
  \item Recursive ORAM
  \end{itemize}    & Implement case studies in \lang;
  show that they are functionally correct and typecheck & 10/07/2020 ($\sim$2 weeks)  \\ \hline
  \rowcolor{implColor}
  Type Checker     & Implement the type checker for \lang                  & 11/07/2020 ($\sim$1 month)  \\ \hline
  \rowcolor{implColor}
  MPC Interpreter  & Implement an EMP MPC backend for \lang                & 01/07/2021 ($\sim$2 months) \\ \hline
  \rowcolor{theoryColor}
  Static Semantics & \lang is \mpc + \obliv                                & 03/07/2021 ($\sim$2 months) \\ \hline
  \rowcolor{theoryColor}
  PMTO\%           & Prove \lang satisfies PMTO\%                          & 05/07/2021 ($\sim$2 months) \\ \hline
  \rowcolor{theoryColor}
  Type Soundness   & Prove \lang satisfies Type Soundness                  & 05/14/2021 ($\sim$1 week)   \\ \hline
  \rowcolor{theoryColor}
  Simulation       & Prove \lang satisfies Simulation                      & 05/21/2021 ($\sim$1 week)   \\ \hline
  Thesis           & Write the thesis and defend it                        & 10/21/2021 ($\sim$5 months) \\ \hline
\end{tabular}

\chapter{Background}
\epigraph{Trust in me in all you do \\
          Have the faith I have in you \\
          Love will see us through \\
          If only you trust in me}{
            Etta James \\
            \emph{Trust in Me}}

For those committe members who aren't (yet) totally jazzed about MPC and ORAM, this one goes out to you.

\section{Secure Multiparty Computation}

The mission of Secure Multiparty Computation (MPC) is to allow many parties who don't trust each
other to compute over their private inputs while only learning the result of the public output.
In this proposal, we will only concern ourselves with the honest-but-curious threat model which
assumes that parties must obey the protocol but may attempt to passively learn information about the secrets of others.

\subsection{Overview}

MPC works by allowing a secret, in cleartext, to be split up into many ``shares'' which are considered ciphertext
and therefore may be safely distributed to other parties and recombined later. More specifically, shares have the following properties:
\begin{enumerate}
\item Shares can be combined to reveal the original cleartext secret.
\item A share does not reveal any information about the secret.
\item Parties can cooperate to compute over shares. For example, being able to create shares of boolean values
  and compute XOR and AND over those shares forms a complete basis for computation. Primitives such as addition,
  comparison, etc. can be built from these boolean operations.
\end{enumerate}

The languages discussed in this proposal are agnostic to the underlying MPC protocol. We only require that the underlying MPC protocol
have the properties listed above. For the purposes of exposition, however, we choose to use the GMW protocol [cite] as a
representative for MPC protocols in general.

In GMW, the secrets being shared are booleans. To represent integers with arithmetic, comparison, etc. we use a two's
complement representation. For example, a digital circuit with only XOR and AND gates can be used to half adders,
full-adders, and ripple-carry adders. A party \alice can generate her share of her (boolean) secret \aliceSec by
generating a random number:

M⁅
\aliceSh{\aliceSec} ← 𝒰(❴0,1❵)
M⁆

The notation ⸨⌊v⌋⸤P⸥⸩ indicates that this is ⸨P⸩'s share of the value ⸨v⸩. Then, \alice generates \bob's share of \aliceSec
as the XOR of her share with the original secret:

M⁅
\bobSh{\aliceSec} ← \aliceSh{\aliceSec} ⊕ \aliceSec
M⁆

At this point, there are a few important things to notice. First, \bob's share is effectively another random number.
As long as he never sees \aliceSh{\aliceSec} he can't distinguish his share \bobSh{\aliceSec} from a fresh, uniform
boolean value. This establishes property (2) of MPC above. Second, XOR has the following properties:

\begin{fact}[⸨⊕⸩-Inverse]
\label{fact:xor-inverse}
  ⸨∀ b ∈ 𝔹␣.␣b ⊕ b = 0⸩
\end{fact}

\begin{fact}[⸨⊕⸩-Identity]
\label{fact:xor-identity}
  ⸨∀ b ∈ 𝔹␣.␣b ⊕ 0 = 0 ⊕ b = b⸩
\end{fact}

These two properties ensure that the original secret, \aliceSec, can be recovered by XOR'ing the shares together:

M⁅
  Aːllll
  A⁅ \aliceSh{\aliceSec} ⊕ \bobSh{\aliceSec} ⧼=⧽ \aliceSh{\aliceSec} ⊕ \aliceSh{\aliceSec} ⊕ \aliceSec & ␠⟅ by \bobSh{\aliceSec} ⟆
  A⁃                                         ⧼=⧽ 0 ⊕ \aliceSec & ␠⟅ by \nameref{fact:xor-inverse} ⟆
  A⁃                                         ⧼=⧽ \aliceSec & ␠⟅ by \nameref{fact:xor-identity} ⟆
  A⁆
M⁆

which establishes MPC property (1) above. Now, let's assume that \bob executes the same protocol to share his secret, \bobSec,
with \alice by splitting it into \aliceSh{\bobSec} and \bobSh{\bobSec}. So, at this point \alice has her shares of both secrets
and similarly for \bob. How can we accomplish property (3) of MPC? To compute ⸨\aliceSec ⊕ \bobSec⸩ we can simply have \alice and \bob
evaluate the XOR of their shares independently:

M⁅
  Aːllll
  A⁅ \aliceSh{A⸨ \aliceSec ⊕ \bobSec A⸩} ⧼←⧽ \aliceSh{\aliceSec} ⊕ \aliceSh{\bobSec}
  A⁃ \bobSh{A⸨ \aliceSec   ⊕ \bobSec A⸩} ⧼←⧽ \bobSh{\aliceSec}   ⊕ \bobSh{\bobSec}
  A⁆
M⁆

Why does this work? Well, it's because XOR is associative:

M⁅
  Aːllll
  A⁅ \aliceSh{\aliceSec ⊕ \bobSec} ⊕ \bobSh{\aliceSec ⊕ \bobSec} ⧼=⧽
      (\aliceSh{\aliceSec} ⊕ \aliceSh{\bobSec}) ⊕ (\bobSh{\aliceSec} ⊕ \bobSh{\bobSec}) & ␠⟪[Definition of XOR of shares]⟫
  A⁃ ⧼=⧽
      (\aliceSh{\aliceSec} ⊕ \bobSh{\aliceSec}) ⊕ (\aliceSh{\bobSec} ⊕ \bobSh{\bobSec}) & ␠⟪[Associativity of XOR]⟫
  A⁃ ⧼=⧽
      \aliceSec ⊕ \bobSec & ␠⟪[Share Recovery]⟫
  A⁆
M⁆

Now for the tricky bit. How do we compute ⸨\aliceSec ∧ \bobSec⸩? To describe this gate we assume that we have access to a
protocol called 1-4 Oblivious Transfer (OT). This protocol allows a sender, ⸨S⸩, to send 4 messages to a receiver, ⸨R⸩, in such a
way that (a) ⸨R⸩ is only allowed to see 1 of the 4 messages and (b) ⸨S⸩ cannot tell which message ⸨R⸩ chose.

Assuming that we have access to such a protocol, we can compute \alice's share of the AND very simply:

M⁅
  Aːllll
  A⁅ \alices{σ}                          ⧼←⧽ 𝒰(❴0,1❵)
  A⁃ \aliceSh{A⸨ \aliceSec ∧ \bobSec A⸩} ⧼←⧽ \alices{σ}
  A⁆
M⁆

Now, we still need to figure out how \bob will compute his share of the AND. This will involve the 1-4 OT protocol in which \alice is the
sender and \bob is the receiver. Consider Table~\ref{tab:and-ot}, which is constructed by \alice. Each row indicates one of the possible
outcomes for \bob's shares, \bobSh{\aliceSec} and \bobSh{\bobSec}.

\begin{table}[h]
  \centering
  \begin{tabular}{|c|c|c|c|}
    \hline
    \bobSh{\aliceSec} & \bobSh{\bobSec} & A⸨ \aliceSh{\aliceSec} ⊕ \bobSh{\aliceSec} ∧ \aliceSh{\bobSec} ⊕ \bobSh{\bobSec} A⸩ & ⸨r⸩ \\ \hline
    ⸨0⸩ & ⸨0⸩ & A⸨ α⸤0,0⸥ = \aliceSh{\aliceSec} ⊕ 0 ∧ \aliceSh{\bobSec} ⊕ 0 A⸩ & A⸨ r⸤0,0⸥ = \alices{σ} ⊕ α⸤0,0⸥ A⸩ \\ \hline
    ⸨0⸩ & ⸨1⸩ & A⸨ α⸤0,1⸥ = \aliceSh{\aliceSec} ⊕ 0 ∧ \aliceSh{\bobSec} ⊕ 1 A⸩ & A⸨ r⸤0,1⸥ = \alices{σ} ⊕ α⸤0,1⸥ A⸩ \\ \hline
    ⸨1⸩ & ⸨0⸩ & A⸨ α⸤1,0⸥ = \aliceSh{\aliceSec} ⊕ 1 ∧ \aliceSh{\bobSec} ⊕ 0 A⸩ & A⸨ r⸤1,0⸥ = \alices{σ} ⊕ α⸤1,0⸥ A⸩ \\ \hline
    ⸨1⸩ & ⸨1⸩ & A⸨ α⸤1,1⸥ = \aliceSh{\aliceSec} ⊕ 1 ∧ \aliceSh{\bobSec} ⊕ 1 A⸩ & A⸨ r⸤1,1⸥ = \alices{σ} ⊕ α⸤1,1⸥ A⸩ \\ \hline
  \end{tabular}
  \caption{B⸨testingB⸩\ins{TODO: couldn't figure out how to use colored math in here without an error.}}
  \label{tab:and-ot}
\end{table}

If \alice now sends ⸨(r⸤0,0⸥,␣r⸤0,1⸥,␣r⸤1,0⸥,␣r⸤1,1⸥)⸩ via 1-4 OT to \bob, then \bob can select the message which corresponds the outcome
of his shares. For example, if \bob's shares are ⸨\bobSh{\aliceSec} = \bobSh{\bobSec} = 0⸩ then he would select ⸨r⸤0,0⸥⸩ (corresponding
to the first row in Table~\ref{tab:and-ot}).

M⁅
  Aːllll
  A⁅ \bobSh{\aliceSec ∧ \bobSec} ⧼←⧽ r␣‹where›␣& r = r⸤0,0⸥␣‹if›␣\bobSh{\aliceSec} = 0␣‹and›␣\bobSh{\bobSec} = 0
  A⁃ & & & r = r⸤0,1⸥␣‹if›␣\bobSh{\aliceSec} = 0␣‹and›␣\bobSh{\bobSec} = 1
  A⁃ & & & r = r⸤1,0⸥␣‹if›␣\bobSh{\aliceSec} = 1␣‹and›␣\bobSh{\bobSec} = 0
  A⁃ & & & r = r⸤1,1⸥␣‹if›␣\bobSh{\aliceSec} = 1␣‹and›␣\bobSh{\bobSec} = 1
  A⁆
M⁆

Finally, let's check that this is correct and secure. First, correctness:

M⁅
  Aːllll
  A⁅ \aliceSh{\aliceSec ∧ \bobSec} ⊕ \bobSh{\aliceSec ∧ \bobSec} ⧼=⧽ \alices{σ} ⊕ (\alices{σ} ⊕ α⸤i,j⸥) & ␠⟪[Definition of ⸨∧⸩]⟫
  A⁃ & & ‹where›␣i = \bobSh{\aliceSec}␣‹and›␣j = \bobSh{\bobSec}
  A⁃ ⧼=⧽ α⸤i,j⸥ & ␠⟪[Fact~\ref{xor-inverse}]⟫
  A⁃ ⧼=⧽ \aliceSh{\aliceSec} ⊕ \bobSh{\aliceSec} ∧ \aliceSh{\bobSec} ⊕ \bobSh{\bobSec} & ␠⟪[Definition (by OT on Table~\ref{tab:and-ot})]⟫
  A⁃ ⧼=⧽ \aliceSec ∧ \bobSec & ␠⟪[Share Recovery]⟫
  A⁆
M⁆

Now, why is this secure? The security relies crucially on the properties of 1-4 OT. If \alice could tell which message \bob chose she would
immediately learn the values of \bob's shares and be able to recover \bob's secret. However, 1-4 OT guarantees that \alice cannot tell
which message \bob chose. Likewise, if \bob were able to see more than one of the messages sent by \alice then ⸨α ⊕ \alices{σ}⸩ would not
sufficiently protect ⸨α⸩\footnote{More formally, the XOR with a random value forms a one-time pad (OTP) encryption scheme, which is only secure
  if the key is never reused.}. However, 1-4 OT guarantees that \bob can only see the message that he chooses.

In the next section, we'll look at how this protocol can be scaled up to a full-blown programming language for secure computation.

\subsection{A Taste of \mpc}

We now have a foundational understanding of how MPC works, but how can we write MPC programs? Symphony is a new
language proposed by Darais et al.~\cite{} which is an expressive, high-level language model for MPC. From here on, we instead refer to this
language instead as \mpc to more clearly constrast it with \obliv which is presented in Chapter~\ref{ch:oram}. As a means of gentle
introduction, let's consider the Millionaire's Problem in which two wealthy parties, \alice and \bob, would like to know who has a higher
net worth.

\begin{figure}[h]
M⁅
\begin{array}{r@{␠}lcl}
   «0:» & 𝑚𝑐3l{ ⦑par⦒[\alice,\bob] }
\\ «1:» & ␠⦑let⦒␣x    ⧼=⧽ ⦑par⦒[\alice]␣⦑read⦒␣⦑in⦒
\\ «2:» & ␠⦑let⦒␣y    ⧼=⧽ ⦑par⦒[\bob]␣⦑read⦒␣⦑in⦒
\\ «3:» & ␠⦑let⦒␣sx   ⧼=⧽ ⦑share⦒[\alice → \alice,\bob]␣x␣⦑in⦒
\\ «4:» & ␠⦑let⦒␣sy   ⧼=⧽ ⦑share⦒[\bob → \alice,\bob]␣y␣⦑in⦒
\\ «5:» & ␠⦑let⦒␣r    ⧼=⧽ sx < sy␣⦑in⦒
\\ «6:» & ␠⦑let⦒␣z    ⧼=⧽ ⦑reveal⦒[\alice,\bob]␣r␣⦑in⦒
\\ «7:» & ␠…
\end{array}
M⁆
\caption{\mpc{} code for the Millionaire's Problem}
\label{fig:millionaires}
\end{figure}

The code in Figure~\ref{fig:millionaires} is written in the \mpc language. On line 0, the ⸨⦑par⦒[\alice,\bob]⸩ block
means that \alice and \bob will execute everything in the block's lexical scope. Any other parties will ignore the
contents of the block and generate an \emph{opaque value} denoted \opaque. On line 1, \alice will ⸨⦑read⦒⸩ her own net worth,
⸨\alices{⋖net-worth⋗}⸩, from her local machine and bind it to ⸨x⸩. \bob, however, will evaluate the expression ⸨⦑par⦒[\alice]␣⦑read⦒⸩ to
\opaque because he is not included in the ⸨⦑par⦒[\alice]⸩ block. On line 2, the same thing happens except \bob reads his net worth,
⸨\bobs{⋖net-worth⋗}⸩, and binds it to ⸨y⸩ while \alice binds \opaque to ⸨y⸩ in her local environment. So, after line 2 the local
environments are as follows.

M⁅
  Aːcc
  A⁅ \alice & \bob
  A⁃ ⟨x ↦ \alices{⋖net-worth⋗},␣y ↦ \opaque⟩␠ & ␠⟨x ↦ \opaque,␣y ↦ \bobs{⋖net-worth⋗}⟩
  A⁆
M⁆

On line 3, \alice splits her ⸨\alices{⋖net-worth⋗}⸩ into two shares. She keeps her share and sends \bob's share to him. We use the same
notation for shares which appears in Section~\ref{sec:mpc} to emphasize the connection between our GMW explanation and the execution model
of \mpc. On line 4, \bob splits his ⸨\bobs{⋖net-worth⋗}⸩ into two shares. He keeps his share and send's \alice's share to her. So, after
line 4 the local environments are as follows.

M⁅
  Aːllll
  A⁅ 𝑚𝑐2c{\alice} & 𝑚𝑐2c{\bob}
    A⁃ ⟨…,&␣sx ↦ \aliceSh{\alices{⋖net-worth⋗}}, & ␠⟨…,&␣sx ↦ \bobSh{\alices{⋖net-worth⋗}}
    A⁃    &␣sy ↦ \aliceSh{\bobs{⋖net-worth⋗}}⟩   &     &␣sy ↦ \bobSh{\bobs{⋖net-worth⋗}}⟩
  A⁆
M⁆

On line 5, \alice and \bob compute over their shares to produce a share indicating who is wealthier. Here we assume that the language knows
how to compute or encode ⸨<⸩ over shares. For example, if the underlying MPC protocol were GMW then we would encode ⸨<⸩ as a digital magnitude
circuit in terms of XOR and AND gates.

M⁅
  Aːcc
  A⁅ \alice & \bob
  A⁃ ⟨…,␣r ↦ \aliceSh{A⸨\alices{⋖net-worth⋗} < \bobs{⋖net-worth⋗}A⸩}⟩ & ␠⟨…,␣r ↦ \bobSh{A⸨\alices{⋖net-worth⋗} < \bobs{⋖net-worth⋗}A⸩}⟩
  A⁆
M⁆

Finally, on line 6, the shares of ⸨\alices{⋖net-worth⋗} < \bobs{⋖net-worth⋗}⸩ among \alice and \bob are combined and the
cleartext result is revealed to both parties. Note that we could have revealed the result only to \alice, for example,
in which case \bob would send his share to \alice but not vice versa. Then, \alice could recover the cleartext result but \bob could not.
In this case, however, they both send their shares to each other. The XOR operator, ⸨⊕⸩, is used to combine shares implicitly as part
of the ⸨⦑reveal⦒⸩.

M⁅
  Aːllll
  A⁅ 𝑚𝑐2c{\alice} & 𝑚𝑐2c{\bob}
    A⁃ ⟨…,␣z &{} ↦ \aliceSh{A⸨\alices{⋖net-worth⋗} < \bobs{⋖net-worth⋗}A⸩} & ␠⟨…,␣z &{} ↦ \aliceSh{A⸨\alices{⋖net-worth⋗} < \bobs{⋖net-worth⋗}A⸩}
    A⁃       &{}␣⊕ \bobSh{A⸨\alices{⋖net-worth⋗} < \bobs{⋖net-worth⋗}A⸩} & &{}␣⊕ \bobSh{A⸨\alices{⋖net-worth⋗} < \bobs{⋖net-worth⋗}A⸩}
    A⁃       &{}␣= \alices{⋖net-worth⋗} < \bobs{⋖net-worth⋗}⟩ & &{}␣= \alices{⋖net-worth⋗} < \bobs{⋖net-worth⋗}⟩
  A⁆
M⁆

There are some interesting things to notice about this program (and \mpc) more generally. First, the programmer controls which parties
evaluate which expressions by using ⸨⦑par⦒⸩ blocks. Second, cleartext ⸨⦑read⦒⸩ operations (lines 1,2) are mixed with ciphertext operations
(lines 3,4,5,6). Last, notice that the original program has a centralized interpretation. In a deployment, this program would run
independently on \alice and \bob who would communicate to jointly compute. Our explanation of the execution of Figure~\ref{fig:millionaires}
did not need to explicate any of that.

\ins{Considering also showing GCD here.}

\section{Oblivious RAM}

\subsection{Overview}

\subsection{A Taste of \obliv}

\chapter{A Language for Concise MPC}

\ins{Take from Symphony paper}

\chapter{A Language for Probabilistically Oblivious Computation}

\ins{Take from lam-obliv paper}

\chapter{Proposal: \lang, A Secure MPC Language With User-Defined ORAM}

An outline.

\begin{enumerate}
\item What? Case Studies. Why? The ORAM implementations presented in \obliv are in client-server model. These implementations will
  be written to abide by the MPC type system and will be in the more restrictive ORAM-SC model. Also, frankly, the existing literature
  on the differences between these two models sucks. I hope to more clearly articulate the differences and similarities in my thesis
  by showing the implementations side-by-side. How? Adapt the implementations from~\citet{lam-obliv} according to the descriptions of
  ORAM-SC in the literature~\cite{}.
\item What? Type Checker. Why? To verify that the formal type sytem is feasible to implement and interesting. By ``interesting'' we mean
  that it can successfully typecheck the ORAM case studies. Also, as an additional contribution, the \mpc implementation presented
  in~\citet{symphony} does not include a type checker. So, we will also show that the interesting case studies from that paper type check.
  Finally, this will allow us to rapidly test different designs for the formal type system. How? Write a bunch of Haskell. As a first cut
  we will graft the affinity and probability region features from \obliv onto the type system of \mpc.
\item What? MPC Interpreter. Why? The formal semantics of \mpc and \lang are parameterized by the underlying MPC protocols. This may
  may create some doubt in readers that the semantics is truly modelling the implementation of an MPC language. Again, the implementation
  will serve to show that the formal semantics are feasible and interesting. How? We have already added the foundation of this functionality
  by using the Haskell FFI to call out to the (two-party, semi-honest) EMP library. For example, we have the Millionaire's Problem working.
\item What? Static Semantics. Why? We enforce the security properties of programs in \lang by showing that security is implied by
  well-typing. We want PMTO\% and simulation, so we have to design a type system which satisfies the former without wrecking the latter.
  How? Having implemented the type checker, we can base the static semantics on that.
\item What? Prove Type Safety, PMTO\%, and Simulation. These are all desirable properties. The first says that the types are coherent with
  the runtime behavior of the program. The second says that programs are secure, they only leak information which is explicitly declassified
  according to a reveal expression. The third says that the single-threaded, or centralized, interpretation of programs is consistent with
  the distributed (``real'') interpretation. How? We expect that the proof of type safety will be a standard syntactic proof using progress
  and preservation. The proof of PMTO\% will build on the techniques (mixed semantics, simulation) for PMTO in \obliv and the proof of MTO\%
  in \mpc. One possible challenge is the proof of simulation between the single-threaded and distributed semantics. Since we are adding a
  probabilistic choice operation to the single-threaded semantics it will no longer be deterministic. This could complicate the proof of
  simulation.
\end{enumerate}

As I mentioned briefly in Chapter~\ref{ch:intro}, the primary goal of this thesis is to develop secure MPC language, \lang, which
support user-defined ORAM implementations. By virtue of all well-typed \lang programs being secure, any well-typed user-defined ORAM
is also guaranteed to be secure. The basis of \lang will be the MPC language, \mpc, presented in Chapters~\ref{ch:taste-mpc},\ref{ch:mpc}.
As we saw in Chapters~\ref{ch:taste-obliv},\ref{ch:obliv}, efficient ORAM implementations rely crucially on the ability to sample from
uniform distributions\footnote{Formally, the logarithmic lower-bound on oblivious simulation only applies in the PRAM model of computation.}.
So, if we want \lang to support efficient, user-defined ORAM constructions we must add to \mpc a feature which permits uniform, random sampling.
While this may seem like a fairly innocuous change, it is veritable Pandora's box with respect to the metatheory of our language. The \mpc
language defines security as~\nameref{thm:mod-mod} which says that memory address traces do not depend on secrets, modulo differences induced
by declassification. After adding uniform, random sampling this property no longer holds. In one case, the property can be violated by programs
which are intuitively safe.

M⁅
\begin{array}{r@{␠}lcl}
   «0:» & 𝑚𝑐3l{ ⦑par⦒[\alice,\bob] }
\\ «1:» & ␠⦑let⦒␣s    ⧼=⧽ ⦑par⦒[\alice]␣⦑read⦒␣⦑in⦒
\\ «2:» & ␠⦑let⦒␣a    ⧼=⧽ ⦑array⦒(2)[λ␣\_ → 0]␣⦑in⦒
\\ «3:» & ␠⦑let⦒␣r    ⧼=⧽ ⦑unif⦒(❴0,1❵)␣⦑in⦒
\\ «4:» & ␠a[r]
\end{array}
M⁆

On line 1, \alice binds a secret value to ⸨s⸩. On line 2, both \alice and \bob declare an array, ⸨a⸩, of two cells both containing the value
⸨0⸩. Finally, on lines 3 and 4, we generate a uniform, random sample from the set ⸨❴0,1❵⸩, bind it to ⸨r⸩ and access the array ⸨a⸩ at index
⸨r⸩. We assume that both \alice and \bob generate the same value when they randomly sample\footnote{This can be achieved by having \alice and \bob generate their own samples, which are then shared and XOR'd with each other.}. This program is cleary secure, since all the code after
line 1 is independent of ⸨s⸩. However, this code does not satisfy~\ref{thm:mto-mod}. This property implicitly assumes that the semantics is
deterministic! In other words, it assumes that any change in the memory address trace must have been induced by dependence on a secret. This
is precisely the issue that \obliv language attempts to overcome by adopting a probabilistic version of MTO,~\ref{thm:pmto}.

\begin{theorem}[PMTO\%]\label{thm:pmto-mod}
\end{theorem}

\chapter{Open Problems}

We briefly discuss some important open problems which affect the design of languages for
MPC and obliviousness. These are not addressed by \lang and I feel that they are promising
areas for future research.

\section{Resource Awareness}

MPC programs are expensive. Something something orders of magnitude slower~\cite{}. This due principally to the communication
required. For example in GMW, an AND gate requires a round of communication and the 1-4 OT requires an encryption scheme.
Programmers who are not experts in MPC will require information about the cost of the programs they are writing. This could be
handled elegantly by the language. There is ample research in type-based resource analysis~\cite{}, for example. \ins{TODO: this is rough}

\section{Oblivious Data Structures}

Our prior work has shown that Oblivious Data Structures (ODS's), proposed by Wang et al., do not satisfy PMTO.\ins{replace with ref. to theorem} Even though ODS's are safe due to negligibl overflow probability, we are unable to verify that fact in \lang. We would like to find some
way to support ODS's within the language while retaining the security guarantees provided by the tyep system.

\section{Other Security Policies}

PMTO\% is perhaps the most obvious security policone could prove about MPC programs which admit ORAM. However, it is not the only property
one could prove. There is a rich literature of different declassification policies, as well as weaker variants of (P)MTO like Differential
MTO~\cite{} and manual computational proofs of securit with explicit complexity-theoretic reductions~\cite{easycrypt}.

\end{document}
