\documentclass{report}

\usepackage{epigraph}
\usepackage{colortbl}
\usepackage{xspace}
\usepackage{listings}
\usepackage{relsize}

\input{darais-latex-imports}
\input{darais-latex-macros}

\newcommand{\lang}{Allyn\xspace}
\newcommand{\mpc}{\ensuremath{\lambda_{\mathrm{MPC}}}\xspace}
\newcommand{\obliv}{\ensuremath{\lambda_{\mathrm{Obliv}}}\xspace}

\newcommand{\ins}[1]{\textcolor{red}{Ian: #1}}

%%% Macros for MPC
\newcommand{\alice}{B⸨‹Alice›B⸩\xspace}
\newcommand{\bob}{C⸨‹Bob›C⸩\xspace}

\newcommand{\alices}[1]{B⸨#1⸤A⸥B⸩}
\newcommand{\bobs}[1]{C⸨#1⸤B⸥C⸩}

\newcommand{\aliceSec}{\alices{s}\xspace}
\newcommand{\bobSec}{\bobs{s}\xspace}

\newcommand{\aliceSh}[1]{\alices{⌊#1⌋}}
\newcommand{\bobSh}[1]{\bobs{⌊#1⌋}}
%%%

\definecolor{implColor}{HTML}{EA9999}
\definecolor{theoryColor}{HTML}{A4C2F4}

\begin{document}

\title{A Programming Language for Obliviousness and Secure Computation}

\author{Ian Sweet \\
  \emph{University of Maryland, College Park} \\
  \emph{ins@cs.umd.edu}}

\date{}

\maketitle

\begin{abstract}
  \emph{Secure Multiparty Computation (MPC)} and \emph{Oblivious RAM} (ORAM) have emerged as promising approaches to
  high-confidentiality computation. Traditionally, MPC programs do not allow secrets to be used as indexes to dynamically
  allocated memory. This precludes, for example, a secure binary search in which the element being accessed is considered secret.
  This restriction is relaxed by using ORAM inside of the secure computation. Unfortunately, existing languages for MPC typically
  support ORAM by ``baking it in'' to the language as a trusted primitive. We propose the design and implementation of a language
  for MPC with \emph{application-level support} for ORAM with an associated \emph{proof of security}.
\end{abstract}

\chapter{Introduction}

\epigraph{It seems like everywhere I go \\
          The more I see, the less I know}{
            Michael Franti \& Spearhead \\
            \emph{Say Hey I Love You}}

If our adversaries were more like Michael Franti then we wouldn't need to concern ourselves with confidentiality.
Unfortunately, that isn't the world we live in. \emph{Secure Multiparty Computation (MPC)} and \emph{Oblivious RAM} (ORAM)
have emerged as promising approaches to high-confidentiality computation. An MPC program allows mutually distrustful
parties to compute over their secret inputs while guaranteeing that they learn nothing more than the result.
Traditionally, these programs do not allow secrets to be used as indexes to dynamically allocated memory.
This precludes, for example, a secure binary search in which the element being accessed is considered secret.
This restriction is relaxed by using ORAM inside of the secure computation. ORAM is an abstract data type with a
RAM-like interface which prevents information leaks through the (physical) memory access pattern side-channel.
So, how do existing MPC languages integrate ORAM? Well, some side-step the issue by relying on applications
to implement ORAM as a library. This has the benefit of empowering the programmer but the detriment of
giving up security guarantees; a buggy ORAM could leak information about secrets accidentally.
Other MPC languages ``bake in'' ORAM as a trusted primitive. This has the converse effect of guaranteeing
security\footnote{Of course, this assumes that the trusted primitive is correct (usually) without a machine-checked proof}
but removing the programmer's ability to choose the ORAM implementation best suited to their application.
What if we could empower programmers to \emph{implement ORAM as a library} and also automatically
\emph{verify that they aren't accidentally leaking their secrets}?

\paragraph{Hypothesis} Motivated by the observation of the previous paragraph, this proposal presents a plan
for the design and implementation of an MPC language, \lang, with \emph{application-level support} for ORAM and an
\emph{automatic proof of security} for all programs written in the language. How will we verify that the hypothesis is correct?
First, we will develop a formal model of \lang and prove that it enjoys security.
Second, we will implement \lang and show that it admits modern ORAM schemes.

\section{Proposed Work}

Tasks in \colorbox{implColor}{red} require adding additional functionality to the existing Haskell implementation of \mpc.
Tasks in \colorbox{theoryColor}{blue} require adapting the formal metatheory of \mpc.

\begin{table}[]
  \begin{tabular}{|p{.33\textwidth}|p{.33\textwidth}|p{.33\textwidth}|}
    \hline
    \textbf{Task} & \textbf{Description} & \textbf{Estimated Time of Completion} \\
    \hline
    \rowcolor{implColor}
    Case Studies:
    \begin{itemize}
    \item Trivial ORAM
    \item Tree ORAM(s)
    \item Recursive ORAM
    \end{itemize}    & Implement case studies in \lang;
                       show that they are functionally correct and typecheck & 10/07/2020 (~2 weeks)  \\ \hline
    \rowcolor{implColor}
    Type Checker     & Implement the type checker for \lang                  & 11/07/2020 (~1 month)  \\ \hline
    \rowcolor{implColor}
    MPC Interpreter  & Implement an EMP MPC backend for \lang                & 01/07/2021 (~2 months) \\ \hline
    \rowcolor{theoryColor}
    Static Semantics & \lang is \mpc + \obliv                                & 03/07/2021 (~2 months) \\ \hline
    \rowcolor{theoryColor}
    PMTO\%           & Prove \lang satisfies PMTO\%                          & 05/07/2021 (~2 months) \\ \hline
    \rowcolor{theoryColor}
    Type Soundness   & Prove \lang satisfies Type Soundness                  & 05/14/2021 (~1 week)   \\ \hline
    \rowcolor{theoryColor}
    Simulation       & Prove \lang satisfies Simulation                      & 05/21/2021 (~1 week)   \\ \hline
  \end{tabular}
\end{table}

\chapter{Background}
\epigraph{Trust in me in all you do \\
          Have the faith I have in you \\
          Love will see us through \\
          If only you trust in me}{
            Etta James \\
            \emph{Trust in Me}}

For those committe members who aren't (yet) totally jazzed about MPC and ORAM, this one goes out to you.

\section{Secure Multiparty Computation}

The mission of Secure Multiparty Computation (MPC) is to allow many parties who don't trust each
other to compute over their private inputs while only learning the result of the public output.
In this proposal, we will only concern ourselves with the honest-but-curious threat model which
assumes that parties must obey the protocol but may attempt to passively learn information about the secrets of others.

\subsection{Overview}

MPC works by allowing a secret, in cleartext, to be split up into many ``shares'' which are considered ciphertext
and therefore may be safely distributed to other parties and recombined later. More specifically, shares have the following properties:
\begin{enumerate}
\item Shares can be combined to reveal the original cleartext secret.
\item A share does not reveal any information about the secret.
\item Parties can cooperate to compute over shares. For example, being able to create shares of boolean values
  and compute XOR and AND over those shares forms a complete basis for computation. Primitives such as addition,
  comparison, etc. can be built from these boolean operations.
\end{enumerate}

The languages discussed in this proposal are agnostic to the underlying MPC protocol. We only require that the underlying MPC protocol
have the properties listed above. For the purposes of exposition, however, we choose to use the GMW protocol [cite] as a
representative for MPC protocols in general.

In GMW, the secrets being shared are booleans. To represent integers with arithmetic, comparison, etc. we use a two's
complement representation. For example, a digital circuit with only XOR and AND gates can be used to half adders,
full-adders, and ripple-carry adders. A party \alice can generate her share of her (boolean) secret \aliceSec by
generating a random number:

M⁅
\aliceSh{\aliceSec} ← 𝒰(❴0,1❵)
M⁆

The notation ⸨⌊v⌋⸤P⸥⸩ indicates that this is ⸨P⸩'s share of the value ⸨v⸩. Then, \alice generates \bob's share of \aliceSec
as the XOR of her share with the original secret:

M⁅
\bobSh{\aliceSec} ← \aliceSh{\aliceSec} ⊕ \aliceSec
M⁆

At this point, there are a few important things to notice. First, \bob's share is effectively another random number.
As long as he never sees \aliceSh{\aliceSec} he can't distinguish his share \bobSh{\aliceSec} from a fresh, uniform
boolean value. This establishes property (2) of MPC above. Second, XOR has the following properties:

\begin{fact}[⸨⊕⸩-Inverse]
\label{xor-inverse}
  ⸨∀ b ∈ 𝔹␣.␣b ⊕ b = 0⸩
\end{fact}

\begin{fact}[⸨⊕⸩-Identity]
\label{xor-identity}
  ⸨∀ b ∈ 𝔹␣.␣b ⊕ 0 = 0 ⊕ b = b⸩
\end{fact}

%M⁅
%   Aːlll
%   A⁅ (1) & ␠∀ b ∈ 𝔹␣.␣b ⊕ b = 0 & ␠⟪[Inverses]⟫
%\\ A⁃ (2) & ␠∀ b ∈ 𝔹␣.␣b ⊕ 0 = 0 ⊕ b = b & ␠⟪[Identity]⟫
%   A⁆
%⁆M

These two properties ensure that the original secret, \aliceSec, can be recovered by XOR'ing the shares together:

M⁅
  Aːllll
  A⁅ \aliceSh{\aliceSec} ⊕ \bobSh{\aliceSec} ⧼=⧽ \aliceSh{\aliceSec} ⊕ \aliceSh{\aliceSec} ⊕ \aliceSec & ␠⟪[Definition of \bobSh{\aliceSec}]⟫ \\
  A⁃                                         ⧼=⧽ 0 ⊕ \aliceSec & ␠⟪[Fact~\ref{xor-inverse}]⟫ \\
  A⁃                                         ⧼=⧽ \aliceSec & ␠⟪[Fact~\ref{xor-identity}]⟫ \\
  A⁆
M⁆

which establishes MPC property (1) above. Now, let's assume that \bob executes the same protocol to share his secret, \bobSec,
with \alice by splitting it into \aliceSh{\bobSec} and \bobSh{\bobSec}. So, at this point \alice has her shares of both secrets
and similarly for \bob. How can we accomplish property (3) of MPC? To compute ⸨\aliceSec ⊕ \bobSec⸩ we can simply have \alice and \bob
evaluate the XOR of their shares independently:

M⁅
  Aːllll
  A⁅ \aliceSh{A⸨ \aliceSec ⊕ \bobSec A⸩} ⧼←⧽ \aliceSh{\aliceSec} ⊕ \aliceSh{\bobSec} \\
  A⁃ \bobSh{A⸨ \aliceSec   ⊕ \bobSec A⸩} ⧼←⧽ \bobSh{\aliceSec}   ⊕ \bobSh{\bobSec} \\
  A⁆
M⁆

Why does this work? Well, it's because XOR is associative:

M⁅
  Aːllll
  A⁅ \aliceSh{\aliceSec ⊕ \bobSec} ⊕ \bobSh{\aliceSec ⊕ \bobSec} ⧼=⧽
      (\aliceSh{\aliceSec} ⊕ \aliceSh{\bobSec}) ⊕ (\bobSh{\aliceSec} ⊕ \bobSh{\bobSec}) & ␠⟪[Definition of XOR of shares]⟫ \\
  A⁃ ⧼=⧽
      (\aliceSh{\aliceSec} ⊕ \bobSh{\aliceSec}) ⊕ (\aliceSh{\bobSec} ⊕ \bobSh{\bobSec}) & ␠⟪[Associativity of XOR]⟫ \\
  A⁃ ⧼=⧽
      \aliceSec ⊕ \bobSec & ␠⟪[Share Recovery]⟫ \\
  A⁆
M⁆

Now for the tricky bit. How do we compute ⸨\aliceSec ∧ \bobSec⸩? To describe this gatewe need to assume that we have access to a
protocol called 1-4 Oblivious Transfer (OT). This protocol allows a sender, ⸨S⸩, to send 4 messages to a receiver, ⸨R⸩, in such a
way that (a) ⸨R⸩ is only allowed to see 1 of the 4 messages and (b) ⸨S⸩ cannot tell which message ⸨R⸩ chose.

Assuming that we have access to such a protocol, we can compute \alice's share of the AND very simply:

M⁅
  Aːllll
  A⁅ \alices{σ}                          ⧼←⧽ 𝒰(❴0,1❵) \\
  A⁃ \aliceSh{A⸨ \aliceSec ∧ \bobSec A⸩} ⧼←⧽ \alices{σ} \\
  A⁆
M⁆

Now, we still need to figure out how \bob will compute his share of the AND. This will involve the 1-4 OT protocol in which \alice is the
sender and \bob is the receiver. Consider Table~\ref{tab:and-ot}, which is constructed by \alice. Each row indicates one of the possible
outcomes for \bob's shares, \bobSh{\aliceSec} and \bobSh{\bobSec}.

\begin{table}[h]
  \centering
  \begin{tabular}{|c|c|c|c|}
    \hline
    \bobSh{\aliceSec} & \bobSh{\bobSec} & A⸨ \aliceSh{\aliceSec} ⊕ \bobSh{\aliceSec} ∧ \aliceSh{\bobSec} ⊕ \bobSh{\bobSec} A⸩ & ⸨r⸩ \\ \hline
    ⸨0⸩ & ⸨0⸩ & A⸨ α⸤0,0⸥ = \aliceSh{\aliceSec} ⊕ 0 ∧ \aliceSh{\bobSec} ⊕ 0 A⸩ & A⸨ r⸤0,0⸥ = \alices{σ} ⊕ α⸤0,0⸥ A⸩ \\ \hline
    ⸨0⸩ & ⸨1⸩ & A⸨ α⸤0,1⸥ = \aliceSh{\aliceSec} ⊕ 0 ∧ \aliceSh{\bobSec} ⊕ 1 A⸩ & A⸨ r⸤0,1⸥ = \alices{σ} ⊕ α⸤0,1⸥ A⸩ \\ \hline
    ⸨1⸩ & ⸨0⸩ & A⸨ α⸤1,0⸥ = \aliceSh{\aliceSec} ⊕ 1 ∧ \aliceSh{\bobSec} ⊕ 0 A⸩ & A⸨ r⸤1,0⸥ = \alices{σ} ⊕ α⸤1,0⸥ A⸩ \\ \hline
    ⸨1⸩ & ⸨1⸩ & A⸨ α⸤1,1⸥ = \aliceSh{\aliceSec} ⊕ 1 ∧ \aliceSh{\bobSec} ⊕ 1 A⸩ & A⸨ r⸤1,1⸥ = \alices{σ} ⊕ α⸤1,1⸥ A⸩ \\ \hline
  \end{tabular}
  \caption{\ins{TODO: couldn't figure out how to use colored math in here without an error.}}
  \label{tab:and-ot}
\end{table}

If \alice now sends ⸨(r⸤0,0⸥,␣r⸤0,1⸥,␣r⸤1,0⸥,␣r⸤1,1⸥)⸩ via 1-4 OT to \bob, then \bob can select the message which corresponds the outcome
of his shares. For example, if \bob's shares are ⸨\bobSh{\aliceSec} = \bobSh{\bobSec} = 0⸩ then he would select ⸨r⸤0,0⸥⸩ (corresponding
to the first row in Table~\ref{tab:and-ot}).



\end{document}
